\documentclass[main.tex]{subfiles}
\begin{document}
%\chapter{Intro}
%\label{chapter:intro}
\centering
\textbf{Thesis Plan}
\begin{enumerate}
    \item Introduction
    \begin{itemize}
        \item Explain the context in which this work is situated and the structure of the thesis
    \end{itemize}
    \item SM and QCD
    \begin{itemize}
        \item introductory content about the SM, non-abelian $SU(3)$ group, QCD Lagrangian
        \item to extract a physical predictions, we need to calculate cross-sections
        \item $\rightarrow$ the need for perturbative expansion -- order by order calculation of amplitudes
        \item loops and divergences, dimensional regularisation, `t Hooft-Veltman scheme
    \end{itemize}
    \item Tools for calculating scattering amplitudes
    \begin{itemize}
        \item general description of the difficulties in these computations, algebraic + analytic complexity, talk about the state of the art developments
        \item Feynman rules
        \item colour decomposition
        \item helicities and helicity amplitudes
        \item spinor-helicity formalism
        \item momentum twistors
        \item finite fields
        \begin{itemize}
            \item mention the connection between finite fields and rational parametrisation of kinematics
            \item mention how the Chinese remainder theorem allows us to reconstruct larger rational numbers by combining several prime fields
        \end{itemize}
        \item multiloop integrand reduction and unitarity cuts - explain how a tensor integral can be reduced onto a combination of scalar integrals - I would like to show a simple example of this
       
        \item IBP reduction
        \begin{itemize}
            \item generation, Laporta algorithm
            \item solution as a linear solver - implemented over FF
            \item UT basis of MIs (maybe talk about MIs in general here and leave the discussion of UT for later sections?):
            \begin{itemize}
                \item what they are, why they are useful
                \item constructing the $\epsilon$-free DEs satisfied by UT integrals\
                \item DEs as a way of solving UT MIs
                \item the `symbol' formalism
            \end{itemize}
            
            \item mention progress in syzygies, intersection numbers?
        \end{itemize}
        
        \item discuss different bases of special functions and their advantages
        \begin{itemize}
            \item analytical and numerical methods for the evaluation of MIs/special functions
            \item describe the theory behind auxiliary mass flow and generalised series expansion?
        \end{itemize}
    \end{itemize}
    
    \item $pp \rightarrow Hb\bar{b}$
    \begin{itemize}
        \item 5-pt 1-mass kinematics: 6 invariants + $\mathrm{tr}_5$
        \item colour structure + describe how the leading-colour approximation simplifies the computation (also mention how (un)justified it is)
        \item reduction onto maximal topologies (show pictures?)
        \item IBPs - describe the planar families

        \item pole subtraction to get finite remainder
        \begin{itemize}
            \item I would like to show how a sample pole structure is derived, but maybe that should be moved to Section 3?
        \end{itemize}
        \item reconstruction tools and complexity
        \begin{itemize}
            \item linear relations
            \item matching onto ansatz
            \item univariate partial-fractioning - mention the impact of the right choice of the variable
        \end{itemize}
        \item describe BCFW to compute tree level amps?
        \item describe checks that are done on the final expressions?
    \end{itemize}
    
    \item $pp \rightarrow W \gamma j$
    \begin{itemize}
        \item mostly same as above, but:
        \begin{itemize}
            \item describe the tensor decomposition method
            \item show how to combine the 5pt and 4pt contributions
        \end{itemize}
        
        \item pentagon functions and why they are useful (closure under permutations of massless legs)
        \item reconstruction - testing of MT parametrisation on 1L expressions - done \textit{a posteriori} to reduce the size of rational coefficients (and also their evaluation time)
    \end{itemize}

    \item Other? Things that could be mentioned, but don't really fit anywhere else:
    \begin{itemize}
        \item reparametrisation of MTs based on one-loop expression testing, implemented \textit{a priori} over FF
        \item at one point, I applied integrand-level symmetries to loop monomials in order to simplify the \texttt{DiagramNumerators}. It helped a bit, but was not crucial in the end.
    \end{itemize}

    \item Conclusions
    \begin{itemize}
        \item a nice summary of our work and how it fits into the wider picture
        \item stress that these results are some of the first at the 2L 5pt 1-mass QCD frontier
        \item outlook
    \end{itemize}
\end{enumerate}
\end{document}