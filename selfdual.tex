\documentclass[main.tex]{subfiles}
\begin{document}
\chapter{The `all-plus' $\phi+\text{gluons}$ amplitudes}
Yet another type of process to which we can apply our technology is the scattering of a Higgs boson with gluons. This process is of high phenomenological interest because the greatest Higgs production mode is the inclusive $gg$ fusion through a quark loop, $gg \rightarrow H + X$. Since the Yukawa coupling of the Higgs boson to quarks is proportional to the quark mass, $gg$ fusion is dominated by the top quark loop specifically. It has been shown that in the limit $2m_t \gg m_H$, the effects of this loop can be reliably captured by integrating it out and adopting an effective vertex of dimension 5, $H G_{\mu\nu} G^{\mu\nu}$ \cite{Wilczek:1977, Djouadi:1991}. This reduces the loop order of the corresponding Feynman diagrams, greatly simplifying the calculations. Two-loop amplitudes for $H+4g$ scattering computed in this approximation contribute not only to NNLO QCD corrections to $pp \rightarrow H+2j$, but also to $pp \rightarrow H+j$ at N3LO and $pp \rightarrow H$ at $N4LO$.
\textcolor{red}{Is this correct? What does $\text{NNLO}_{HTL} \otimes \text{NLO}_{QCD}$ in Les Houches mean?}

Unfortunately, due to the large symmetry of the external particles \textcolor{red}{How to express this better?}, this process is significantly more computationally demanding than the processes considered in the previous chapters. In particular, we have to handle a far greater number of integral topologies and non-planar families appear already at leading-colour \textcolor{red}{check if this was true, I'm not sure now}. At the time of writing, a UT master integral basis for the non-planar double-pentagons and the corresponding expansion into special functions are not available in literature yet. We can, however, study the 4-point amplitude in anticipation of these results. In this chapter, we give a brief overview of the computation of the $H+3g$ amplitude in the `all-plus' helicity configuration using the so-called `self-dual' model. We also present the results for the cut-constructible part of this amplitude obtained using \textcolor{red}{D-dimensional?} unitarity cuts. Finally, we introduce the `symbol' formalism, which is closely related to the differential equations and UT master integrals, and describe the pole structure of this amplitude.

\section{The self-dual model}
Following Ref.~\cite{}
\end{document}