\documentclass[main.tex]{subfiles}
\begin{document}
\chapter{Conclusions} \label{sec:conclusions}
In this thesis, we presented selected techniques needed for precision calculations of high-multiplicity loop scattering amplitudes. Such calculations constitute a crucial ingredient of the LHC precision programme, which aims to bring the theoretical uncertainties to the percent level and below. We started by discussing the general framework for computing predictions for physical observables using the machinery of QFT. Having established the key role of the scattering amplitude in this process, we described the challenges associated with their computation in practice. We dedicated Chapter~\ref{sec:tools} to a detailed discussion of a variety of tools employed to overcome these challenges. In particular, we showed how the use of finite fields helps us tackle the daunting algebraic complexity present in state-of-the-art calculations. We also noted the importance of the IBP reduction, which hugely reduces the number of integrals that need to be computed for the amplitude. This led us to the technique of differential equations, and in particular differential equations in the canonical form, which can be used to evaluate bases of pure master integrals with unexpected ease.

We applied the techniques presented in the first part of this thesis to three processes at the cutting edge of QCD and QED computations. First, we computed the two-loop helicity amplitudes for the production of the Higgs boson in association with a bottom-quark pair. This five-point process with a massive external leg is of great phenomenological interest, since an improved measurement of the bottom-quark Yukawa coupling could help constrain certain supersymmetric models which affect this coupling. In order to speed up the reconstruction of the analytic form of the rational coefficients from their numerical samples over finite fields, we implemented several optimisation tools which lower the complexity of this task. Moreover, exploiting the differential equation technique, we constructed a special function basis for the finite remainder of the amplitude which can be efficiently evaluated across the full phase space.

The second computation we tackled was the production of the $W$ boson in association with a photon and a jet. Here, the $W$ boson was assumed to further decay into a lepton pair, which allowed us to apply the familiar five-point kinematics with one off-shell leg to the scattering of six massless particles. Studying $W\gamma + n (\text{jet})$ production grants access to the $W W \gamma$ coupling, which can also be affected by new physics scenarios. In this work, we employed a new technique for simplifying the reconstructed functional coefficients based on a systematic search for the optimal rational parametrisation. In the future, this technique can be readily implemented also before reconstruction as an additional node of finite field operations, such that the search will not suffer from the algebraic complexity. Finally, in contrast to the previous set of amplitudes, we employed the tailored set of one-mass pentagon functions for the expansion of master integrals. These functions have the key advantage of being closed under the permutations of the massless momenta, which allows us to significantly reduce the number of kinematic points at which they have to be evaluated in order to serve a phenomenologically useful role.

Finally, we switched our focus to QED, computing the two-loop amplitudes for the scattering of a lepton pair with an off-shell and an on-shell photon. Our work completes the amplitude-level ingredients contributing to the N$^3$LO predictions of electron-muon scattering $e \mu \to e \mu$, which are required to meet the precision goal of the MUonE experiment. Here, the lower multiplicity meant that the reconstruction of analytic expressions from finite fields did not pose as serious a challenge as for the previous two processes. Instead, we focused on optimising the setup of the IBP reduction by exploiting numerical permutations of the IBP solutions obtained within a small set of families with specific orderings of external legs. This method allowed us to decrease the time and memory consumption of the IBP reduction stage, which becomes especially important in the case of processes with many integral families. Furthermore, we have constructed a basis of algebraically independent GPLs needed for the master integrals relevant to any scattering process of four massless particles with a single external off-shell leg up to two loops. This guarantees an efficient numerical evaluation and a compact analytic representation of the amplitudes.

Overall, we believe that the work presented in this thesis represents a valuable contribution to the study of scattering amplitudes. We hope that the detailed introductory material serves as a useful reference for those new to the subject. With more and more data pouring in from the LHC and future colliders, it is clear that further breakthroughs in our computational techniques are needed to fully control the NNLO frontier and move towards the N$^3$LO one. The Les Houches wishlist~\cite{Huss:2022ful}continuously calls for new theoretical results, of which the scattering amplitudes are a key ingredient. In particular, we point out the recent completion of the pure master integral bases for all two-loop integrals for five-particle kinematics with an off-shell leg. This opens the door to a new range of computations, such as $pp \rightarrow H+2j$ and $pp \rightarrow W \gamma \gamma $ at NNLO, as well as computing the sub-leading colour corrections to the QCD processes we covered in this thesis.
\end{document}