\documentclass[main.tex]{subfiles}

\begin{document}
\begin{enumerate}
    \item \textbf{Chapter 3. Put the computation in perspective from the experimental point of view} \\
    \textcolor{gray}{2005.10277} 
    \\
    At the LHC, in the case of the bottom-quark Yukawa coupling ($y_b$), a direct sensitivity can
be in principle achieved both via the $H \rightarrow \bar{b}b$ decay or via the associate production of a Higgs
boson together with a $\bar{b}b$ pair. While the $H \rightarrow \bar{b}b$ decay has already been observed in conjunction
with V H production [5, 6, 17–20], no dedicated SM analysis has ever been performed by the
ATLAS and CMS collaboration in order to measure the $H\bar{b}b$ production process. Indeed, this
process has an inclusive cross section that is comparable to the one of $H \bar{t}t$ production (e.g. ∼ 0.5
pb at 13 TeV). However, at variance with $H \bar{t}t$ production, at least one b-jet has to be tagged
in order to make it distinguishable from the inclusive Higgs-boson production process, whose
production rate is ∼ 100 times larger than $H\bar{b}b$ production alone. The problem is that tagging
a b-jet dramatically reduces the cross section, but as said this is an unavoidable procedure for
obtaining a possible direct sensitivity on the bottom-quark Yukawa coupling. Without tagging
any b-jet, even bottom-quark loops in ggF, which induce an indirect sensitivity on $y_b$, have a
larger contribution: about $\approx6\%$ for the inclusive cross section and up to $\approx10\%$ for the Higgs
boson at small transverse momentum [21–28].
\\
n our study, we demonstrate that the idea of directly extracting $y_b$ from the measurement of
$H\bar{b}b$  at the LHC is substantially hopeless. The rates for this process are small and contaminated
by terms that depend on $y_t$ and the $HZZ$ coupling. Reducing this contamination implies also
a strong reduction of the cross section of the term depending only on $y_b$.
\\
\textcolor{gray}{2011.13945}
\\
Despite the large
branching ratio of the Higgs decaying to the bottom pair, the $H\bar{b}b$ production eluded measurement
until much later through the $V h, h \rightarrow \bar{b}b$ channel [6–8], bringing about sensitivity to the
bottom-quark Yukawa, $y_b$, as well. The current sensitivity on $\kappa_b \propto y{b}/y_{SM}$
b is of the order
of $7\%$ and it is expected to improve to $2.2\%$ at HL-LHC [9]. Other direct probes to the
Higgs couplings to the bottom quark can be envisaged through the measurements of the
associated production of the Higgs with the bottom quark in the associated $bh$ or $\bar{b}bh$
production channels.
The $\bar{b}bh$ signal is sizable at the LHC with a cross-section of about $0.5 pb$. Calculations of the $\bar{b}bh$ cross-section focusing on the contribution proportional to $y_b$ has been a
developing effort, with improvements from including higher order contribution, matching
and resumming bottom-mass effects, evaluation of four and five-flavor matching scheme
and optimal scale choices for Monte Carlo implementation with parton showers [10–21].
Despite the large inclusive rate, $\bar{b}bh$ production remains a challenging channel for experimental measurements. Given that the bulk of the signal contains only soft $p_T$ $b$-jets, which
evade selection cuts, the signal drops by orders of magnitude after two, or even one, $b$-jets
are explicitly tagged. It is even more challenging to achieve sensitivity on the $y_b$ coupling
among all the amplitudes contributing to $\bar{b}bh$ production. Even with the large amount of
luminosity that will be available at HL-LHC, the $y_b$ mediated diagrams contribute only a
small fraction of the total cross section to the inclusive rate [20, 21].
In this work we shall appeal to kinematic shapes, machine learning and game theory to
glean information on $y_b$ from the associated production of the Higgs with a bottom-quark
pair. In doing so we would like to clarify the following open questions:
\\
\textcolor{gray}{2101.10683}
\\
Most importantly, its interaction strengths with relatively
heavy fermions are not yet known precisely enough, in contrast to the interaction
with gauge boson pairs, where the uncertainty is much lesser. \\
\hrule
From the perspective of phenomenology, $\ppbbh$ production at the LHC has been a subject of great interest due to its
potential in directly measuring the bottom-quark Yukawa coupling $y_b$.
In the SM, the coupling strengths of the Higgs boson to the fermions and vector bosons are proportional to their mass, causing the rate of the $b\bar{b}H$ production to be suppressed with respect to, for example,
Higgs production in gluon fusion ($gg\to H$) or vector boson fusion ($pp\to Hjj$), associated
production with a vector boson ($pp\to VH$), and associated production with a top-quark pair ($pp\to
t\bar{t}H$). 
\textcolor{red}{The inclusive $b \bar{b} H$ production cross section is $\approx0.5$ pb at $13$ TeV, however at least one $b$-jet needs to be tagged to distinguish this process from inclusive Higgs boson production, which lowers the detected rate by orders of magnitude.}

\textcolor{red}{The ratio of $y_b$ in a new physics scenario to $y_b$ in the SM, $\kappa_b = y_b / y_b^{\text{SM}}$, is currently known to \textbf{7\% / 2\% ???} and will be measured more precisely at high-luminosity LHC~\cite{Cepeda:2019klc, 2019369}}. In some scenarios, such as the Two Higgs Doublet Models (2HDM's) and the Minimal
Supersymmetric Standard Model (MSSM), the bottom-quark Yukawa coupling can be dramatically enhanced,
resulting in a considerable increase of the $b\bar{b}H$ production
rate~\cite{Balazs:1998nt,Dawson:2005vi}.  Thus, the study of the $b\bar{b}H$ production will allow
to constrain supersymmetric models and other extensions of the SM that modify the bottom-quark
Yukawa coupling. Recent studies on the interplay between $b\bar{b}H$ signal and backgrounds can be
found in Refs.~\cite{Pagani:2020rsg,Grojean:2020ech,Konar:2021nkk}.
\item \textbf{Pg. 68 has the sentence “The presence of b-tagging further suppresses the bbH
production rate” Though the meaning can be reconstructed, the presence or not of
tagging does not change a rate, it changes detection}
See above - `suppresses the production rate' $\rightarrow$ `lowers the detected rate'

\item \textbf{Summary of Chapter 3, sentence “To the best of our knowledge this is the first…”
can be broken down and clarified to further highlight the novelty of the result}
\\
Quote 5-pt massless and 4-pt 1-mass results?

\item \textbf{Chapter 4, again put in context, e.g. there seems to be one loop EW corrections
that might be more relevant than those computed, are these known?}

\item \textbf{Given the title contains “high multiplicity”, maybe comment on the Summary what
would take to tackle higher-legs amplitude and how the “amount of work” would
scale}
\begin{itemize}
    \item 8 kinematic invariants even in the massless case
    \item In terms of the IBP relations, while the three-loop, four-point kinematics with up to one off-shell leg has received significant attention already (cite papers), the IBPs for the six-point case remain unknown even at two loops and in a fully massless configuration
    \item While we should be careful not to take the number of Feynman diagrams as a definitive measure of complexity, we can obtain its rough estimate by looking at how the number of diagrams scales as we add more external legs to a given process. For example, adding an extra gluon to the $\bbqqh$ channel of $pp\to b\bar{b}H$ production in Chapter~\ref{sec:Hbb} increases the number of diagrams from 720 to 10142. Similarly, for the $\bbggh$ channel, the number grows from 3690 to 57478.
    \item As a preliminary estimate, we observe that the inclusion of non-planar families increases the number of integrals included in the Laporta IBP system by a full order of magnitude. The accompanying increase in reconstruction time is approximately \textcolor{red}{insert number}.
    We remark that this problem can be mitigated by using IBP relations generated using so-called syzygy relations \textcolor{red}{insert the usual syzygy papers}. This technique shows great promise in reducing the time of the IBP reduction stage by up to an order of magnitude. We refer the reader to Ref.~\cite{NeatIBP} \textcolor{red}{reference to NeatIBP}
\end{itemize}
\end{enumerate}
\end{document}