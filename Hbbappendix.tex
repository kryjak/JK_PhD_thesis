\documentclass[main.tex]{subfiles}

\begin{document}
\renewcommand{\chaptername}{}
\renewcommand{\thechapter}{}
\renewcommand{\thetable}{A.\arabic{table}}
\renewcommand{\thesection}{A.\arabic{section}}
\renewcommand{\thesubsection}{A.\arabic{subsection}}
\renewcommand{\theequation}{A.\arabic{section}.\arabic{equation}}
\chapter{Appendix A}

\section{Renormalisation Constants} \label{app:renormconstants}
The one- and two-loop bottom-quark Yukawa renormalisation constants entering the UV counterterm in 
Eqs.~\eqref{eq:poles1L}~and~\eqref{eq:poles2L} are~\cite{Ahmed:2014pka,Mondini:2021nck}
\begin{align}
s_1 & = -\frac{3 C_F}{2\eps} \,, \\
s_2 & = \frac{3}{8\eps^2}\big(3 C_F^2 + \beta_0 C_F \big) - \frac{1}{8\eps} \bigg( \frac{3 C_F^2}{2} + \frac{97}{6} C_F C_A -\frac{10}{3} C_F T_F n_f   \bigg) \,,
\end{align}
while the $\beta$ function coefficients and anomalous dimensions are~\cite{Becher:2009qa}

\begingroup
\allowdisplaybreaks
\begin{align}
\beta_0 = & \;\frac{11}{3}C_A - \frac{4}{3} T_F n_f \,, \\
\beta_1 = & \; \frac{34}{3} C_A^2 - \frac{20}{3} C_A T_F n_f - 4 C_F T_F n_f \,, \\
\gamma_0^g = & \; -\frac{11}{3}C_A + \frac{4}{3} T_F n_f \,,   \\
\gamma_1^g = & \;  C_A^2 \left( -\frac{692}{27} + \frac{11\pi^2}{18} + 2 \zeta_3\right)
               + 4 C_F T_F n_f
               + C_A T_F n_f \left( \frac{256}{27} - \frac{2\pi^2}{9}\right) \,, \\
\gamma_0^q = & \; -3 C_F \,, \\
\gamma_1^q = & \; C_F^2 \left( -\frac{3}{2} + 2 \pi^2 - 24 \zeta_3 \right)
                  + C_F C_A \left( -\frac{961}{54} -\frac{11\pi^2}{6} + 26 \zeta_3 \right) \nn
             & \;  + C_F T_F n_f \left( \frac{130}{27} + \frac{2\pi^2}{3} \right) \,, \\
\gamma_0^\cusp = & \; 4 \,, \\
\gamma_1^\cusp = & \; \left( \frac{268}{9} - \frac{4\pi^2}{3} \right) C_A -\frac{80}{9} T_F n_f \,,
\end{align}
\endgroup
where $C_A = N_c$, $C_F = (N_c^2-1)/(2N_c)$ and $T_F = 1/2$.

\section{One-Loop Results}
\label{app:oneloop}
\begin{table}[t!]
\centering
\begin{tabularx}{1.0\textwidth}{|C{0.7}|C{0.8}|C{0.5}|C{1.2}|C{1.2}|C{1.3}|C{1.3}|}
\hline
 $\bbggh$     & helicity & $\eps^{-2}$ & $\eps^{-1}$ & $\eps^{0}$ & $\eps^{1}$ & $\eps^{2}$ \\
\hline
$\hat A^{(1),1}$ & $++++$ & $ -3 $ & $ 3.07857 - 3.14159 i$ & $ 0.317351 + 8.42128 i$ & $ -1.25257 - 8.56907 i$ & $ 25.8294 - 4.35648 i $ \\
                 & $+++-$ & $ -3 $ & $ 3.07857 - 3.14159 i$ & $ -2.99786 - 1.02133 i$ & $ 2.86487 - 28.7164 i$ & $ 30.3093 - 26.3373 i $ \\
                 & $++-+$ & $ -3 $ & $ 3.07857 - 3.14159 i$ & $ -0.119814 + 8.67497 i$ & $ -1.43041 - 5.33656 i$ & $ 19.6373 - 0.110475 i $ \\
                 & $++--$ & $ -3 $ & $ 3.07857 - 3.14159 i$ & $ -6.51606 + 18.6156 i$ & $ -21.7849 + 24.0036 i$ & $ -24.6605 + 55.6878 i $ \\
\hline
$\hat A^{(1),n_f}$ & $++++$ & $ 0$ & $ 0$ & $ -0.005010 + 0.000779871 i$ & $ -0.00700827 - 0.0150298 i$ & $ 0.0109029 - 0.0163643 i $ \\
                   & $+++-$ & $ 0$ & $ 0$ & $ 0$ & $ 0$ & $ 0 $ \\
                   & $++-+$ & $ 0$ & $ 0$ & $ 0$ & $ 0$ & $ 0 $ \\
                   & $++--$ & $ 0$ & $ 0$ & $ -0.393552 + 0.138515 i$ & $ -0.793221 - 1.11035 i$ & $ 0.635641 - 1.48796 i $ \\
\hline
$\bbqqh$     & helicity & $\eps^{-2}$ & $\eps^{-1}$ & $\eps^{0}$ & $\eps^{1}$ & $\eps^{2}$ \\
\hline
$\hat A^{(1),1}$ & $+++-$ & $ -2 $ & $ 2.48840$ & $ -9.99430 - 8.95182 i$ & $ 2.20899 - 24.3401 i$ & $ 4.76962 - 27.6604 i $ \\
                 & $++-+$ & $ -2 $ & $ 2.48840$ & $ -8.43825 - 7.45006 i$ & $ 7.21741 - 24.6383 i$ & $ 13.6369 - 20.4876 i $ \\
\hline
$\hat A^{(1),n_f}$ & $+++-$ & $ 0$ & $ -0.666667$ & $ 0.726782 - 2.09440 i$ & $ 2.29387 + 2.28325 i$ & $ -2.54017 + 0.316127 i $ \\
                   & $++-+$ & $ 0$ & $ -0.666667$ & $ 0.726782 - 2.09440 i$ & $ 2.29387 + 2.28325 i$ & $ -2.54017 + 0.316127 i $ \\
\hline
\end{tabularx}
\caption{\label{tab:benchmarkbare1L} Numerical values of the bare $\bbggh$ and $\bbqqh$ partial amplitudes at one loop (normalised to the tree-level amplitude) at the kinematic point in 
Eq.~\eqref{eq:physicalpointHbbMomTwistor} for the four independent helicity configurations and the various closed fermion loops contributions. }
\end{table}

\begin{table}[t!]
\centering
\begin{tabularx}{0.85\textwidth}{|C{0.7}|C{1.1}|C{1.1}|C{1.1}|}
\hline
channel & $\mathcal{H}^{(0)}$ & $\mathrm{Re}\;\mathcal{H}^{(1),1}$ & $\mathrm{Re}\;\mathcal{H}^{(1),n_f}$  \\
\hline
$\mathbf{gg}$       & $ 1121.375369 $    & $ 4905.689964$ & $ 204.1069797 $ \\
$\mathbf{q\bar{q}}$ & $ 0.001095232986 $ & $ -0.008958148524$ & $ 0.0007959961305 $ \\
$\mathbf{\bar{q}q}$ & $ 0.001095232986 $ & $ 0.01182947634$ & $ 0.0007959961305 $ \\
$\mathbf{b\bar{b}}$ & $ 738.4111805 $ & $ 5948.275150$ & $ -2005.976183 $ \\
$\mathbf{\bar{b}b}$ & $ 774.9861507 $ & $ -8346.007933$ & $ -2253.325645 $ \\
$\mathbf{bb}/\mathbf{\bar{b}\bar{b}}$ & $ 71.81424881 $  & $ -678.1382010$ & $ -243.5040325 $ \\
\hline
\end{tabularx}
\caption{\label{tab:benchmarkfinremsq1L} Numerical values of the tree-level reduced squared amplitudes  $\mathcal{H}^{(0)}$ and 
one-loop reduced squared finite remainders $\mathcal{H}^{(1)}$ 
defined in Eqs.~\eqref{eq:channel_gg}-\eqref{eq:channel_BB} at the kinematic point in 
Eq.~\eqref{eq:physicalpointHbbMomTwistor} for  the closed fermion loops contributions and the scattering channels specified in Eq.~\eqref{eq:channel_definition}.}
\end{table}

We present in Table~\ref{tab:benchmarkbare1L} the numerical values of the one-loop bare helicity amplitudes at the
kinematic point given in Eq.~\eqref{eq:physicalpointHbbMomTwistor}, evaluated through $O(\eps^2)$ for the different closed fermion loop contributions defined in Eq.~\eqref{eq:NfDecomposition1L}. 
These one-loop amplitudes are required in the computation of the two-loop pole terms in Eqs.~\eqref{eq:poles1L}~and~\eqref{eq:poles2L}. 
Furthermore, Table~\ref{tab:benchmarkfinremsq1L} shows the tree-level reduced squared amplitudes and
one-loop reduced squared finite remainders for the scattering channels listed in Eq.~\eqref{eq:channel_definition}, at the same kinematic point.

\end{document}