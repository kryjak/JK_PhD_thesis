\documentclass[main.tex]{subfiles}

\begin{document}
\renewcommand{\chaptername}{}
\renewcommand{\thechapter}{}
\renewcommand{\thetable}{C.\arabic{table}}
\renewcommand{\thesection}{C.\arabic{section}}
\renewcommand{\thesubsection}{C.\arabic{subsection}}
\renewcommand{\theequation}{C.\arabic{section}.\arabic{equation}}
\chapter{Appendix C} \label{app:polestructure}
In the following, we provide a detailed derivation of the full pole structure of the $\ppbbh$ amplitudes presented in Eqs.~\ref{eq:poles1L} to \ref{eq:H2bbggH}. We do this for completeness, but also in the hope of illuminating the procedure, which can be far from obvious. 
\section{UV singularities}
Let us start with the UV singularities. Renormalising the amplitude amounts to replacing the bare couplings in the Lagrangian with the physical ones according to:
\begin{equation}
    x_B = Z_x x_R \,.
\end{equation}
The renormalisation factors are expanded as $Z_x = 1 + \delta_x$, where $\delta_x$ are the counterterms added to the bare Lagrangian and which are designed precisely to cancel out the divergences. Thus, we can think of renormalisation as absorbing the infinities from $x_B$ into $Z_x$. Then, working with the renormalised parameters $x_R$ in the Lagrangian is known as `renormalised perturbation theory'. Let us now see how to achieve that. Specifying to the case of the $\ppbbh$ process in Sec.\ref{sec:Hbb}, the bare amplitude admits the following expansion (similar to that of Eq.~\ref{eq:ampexpansion})\footnote{See Refs.~\cite{Ahmed:2014pka, Mondini:2019vub} for details, but note the difference of $\alpha_{s,B}^{1/2}$ as the leading power due to one fewer jet and, in the second reference, the expansion in powers of $\alpha_{s,B}/(2 \pi)$, which introduces relative factors of $2$ with respect to our convention}:
\begin{equation} \label{eq:ampexpbare}
    A_B = a_B y_{b,B} \left(A_B^{(0)} + a_B A_B^{(1)} + a_B^2 A_B^{(2)} + \ldots \right)\,,
\end{equation}
where $a_B = \alpha_{s,B}/(4\pi)$ and $\alpha_{s,B}\,$,  $y_{b,B}$ are the bare strong coupling constant and the bare Yukawa coupling of the $b$ quark\JK{What about the renormalisation of $m_B$ and the wavefunction?}. We will work in the $\overline{\text{MS}}$ scheme and make the following replacements for the two couplings:
\begin{align}
    \alpha_{s,B} &= S_\eps\, Z_{\alpha_s} \, \alpha_{s,R}(\mu_R) \label{eq:baretorenormalpha} \\
    y_{b,B} &= Z_y \, y_{b,R}(\mu_R) \label{eq:baretorenormyb} \,,
\end{align}
where $\mu_R$ is the renormalisation scale. The factor $S_\eps = e^{\eps \gamma_E} (4\pi)^{-\eps}$ cancels out with the $m_\eps$ factor extracted from the colour-ordered amplitudes in Eq.~\ref{eq:colourdecomposition} and so will be dropped in the following discussion. The renormalisation factors themselves admit a perturbative expansion in the renormalised strong coupling constant:
\begin{align}
    Z_{\alpha_s} &= 1 + a_R r_1 + a_R^2 r_2 + \order{a_R^3}  \label{eq:Zalphacoeff} \\
    Z_y &= 1 + a_R s_1 + a_R^2 s_2 + \order{a_R^3} \label{eq:Zybcoeff}\,,
\end{align}
where $a_R = \alpha_{s,R}/(4\pi)$. Substituting Eqs.~\ref{eq:baretorenormalpha} through \ref{eq:Zybcoeff} into Eq.~\ref{eq:ampexpbare}, we obtain the amplitude expansion in terms of the renormalised parameters:
\begin{equation} \label{eq:ampexprenorm}
    A_R = a_R y_{b,R} \left(A_R^{(0)} + a_R A_R^{(1)} + a_B^2 A_R^{(2)} + \ldots \right)\,,
\end{equation}
with the individual amplitudes given by:
\begin{align}
    A_R^{(0)} &= A_B^{(0)} \,,\\ \nonumber
    A_R^{(1)} &= A_B^{(1)} + 2 A_B^{(0)}(r_1 + s_1) \,, \label{eq:amprenorminbare}\\ 
    A_R^{(2)} &= A_B^{(2)} + 2 A_B^{(1)}(2r_1 + s_1) + 4A_B^{(0)} (r_2 + r_1 s_1 + s_2)\,. \nonumber
\end{align}
The coefficients $r_1, r_2, s_1, s_2$ are listed in Appendix~\ref{app:renormconstants}. Note that we use their values as given in Ref.~\cite{Mondini:2019vub} for the sake of consistency, but in Eq.~\ref{eq:amprenorminbare} we have rescaled them by powers of 2 to make them compatible with our expansion). Overall, we see that each UV-finite amplitude $A_R^{(L)}$ is defined by adding to the bare amplitude $A_B^{(L)}$ terms related to the $\beta$-function whose $\eps$ poles cancel out the UV divergences. This completes the renormalisation of the$\ppbbh$ amplitudes in Chapter~\ref{sec:Hbb}.
\section{IR singularities}
The renormalised amplitudes $A_R^{(L)}$ are only UV-finite --- the IR divergences remain. As explained in Sec.~\ref{sec:IRdivergencesandKLN}, the IR pole structure of two-loop amplitudes in massless gauge theories was originally derived in Ref.~\cite{Catani:1998bh} and later extended in  Refs.~\cite{Becher:2009cu, Becher:2009qa, Gardi:2009qi}. Here, we provide a brief summary of these results, followed be their application to our process of interest, $\ppbbh$.
\subsubsection{Overview of IR singularities}
The $\overline{\text{MS}}$-renormalised amplitudes $A_R^{(L)}$ admit the following structure:
\begin{align}
    A_R^{(1)} &= \bm{I}^{(1)}(\eps) A_R^{(0)} + F^{(1)}\,, \\
    A_R^{(2)} &= \bm{I}^{(1)}(\eps) A_R^{(1)} + \bm{I}^{(2)}(\eps) A_R^{(0)} + F^{(2)}\,.
\end{align}
Thus, the IR-divergent part is determined by amplitudes at lower loop order, while the full amplitude also receives a finite part $F^{(L)}$, which is the genuinely new contribution that needs to be computed. The pole operators are given by:
\begin{align}
    \bm{I}^{(1)}(\eps) &= \frac{1}{2} \frac{e^{\eps \gamma_E}}{\Gamma(1-\eps)} \sum_i \left(\frac{1}{\eps^2} - \frac{\gamma_0^i}{2\eps} \frac{1}{\bm{T}_i^2}\right) \sum_{j\neq i} \bm{T}_j \cdot \bm{T}_j \left(\frac{\mu_R^2}{-s_{ij}} \right)^\eps\,, \\
    \bm{I}^{(2)}(\eps) &= \frac{e^{-\eps \gamma_E} \Gamma(1-2\eps)}{\Gamma(1-\eps} \left( \frac{\gamma_1^\text{cusp}}{8} + \frac{\beta_0}{2\eps} \right) \bm{I}^{(1)}(2\eps) - \frac{1}{2}\bm{I}^{(1)}(\eps) \left(\bm{I}^{(1)}(2\eps) + \frac{\beta_0}{\eps} \right) + \bm{H}_{\text{RS}}^{(2)}(\eps)\,,
\end{align}
where $s_{ij} = 2\sigma_{ij} p_i \cdot p_j$ with $\sigma_{ij}=+1$ if both momenta are incoming or outgoing and $\sigma_{ij}=-1$ otherwise. The values of the $\beta$-function coefficients and the anomalous dimensions are given in Appendix~\ref{app:renormconstants}. We remark that the one-loop pole operator $\bm{I}^{(1)}(\eps)$ is independent of the regularisation scheme, while the two-loop operator $\bm{I}^{(2)}(\eps)$ is not. Its dependence enters through the quantity $\bm{H}_{\text{RS}}^{(2)}(\eps)$, which contains $\order(1/\eps)$ poles only:
\begin{align}
    \bm{H}_{\text{RS}}^{(2)}(\eps) &= \frac{1}{16\eps} \sum_i \left(\gamma_1^i - \frac{1}{4} \gamma_1^{\text{cusp}}\gamma_0^i + \frac{\pi^2}{16} \beta_0 \gamma_0^{\text{cusp}} C_i \right) \\
    &+ \frac{i f^{abc}}{24\eps} \sum_{(i,j,k)} \bm{T}_i^a \bm{T}_j^b \bm{T}_k^c \ln\left(\frac{-s_{ij}}{-s_{jk}} \right) \ln\left(\frac{-s_{jk}}{-s_{kj}} \right) \ln\left(\frac{-s_{ki}}{-s_{ij}} \right) \\
    &-\frac{i f^{abc}}{128\eps} \gamma_0^{\text{cusp}} \sum_{(i,j,k)} \bm{T}_i^a \bm{T}_j^b \bm{T}_k^c \left( \frac{\gamma_0^i}{C_i} - \frac{\gamma_0^j}{C_j} \right)\ln\left(\frac{-s_{ij}}{-s_{jk}} \right) \ln\left(\frac{-s_{ki}}{-s_{ij}} \right)\,,
\end{align}
where the sum runs over unordered tuples $(i,j,k)$ of distinct parton indices. As pointed out in Ref.~\cite{Becher:2009cu}, the last two lines appear only because the pole operators $\bm{I}^{(L)}(\eps)$ formulated in Ref.~\cite{Catani:1998bh} were not defined in a minimal scheme, but also include terms finite in $\eps$. Due to colour conservation, they contribute only if the amplitude contains at least four partons \JK{The last line appears only for \textit{more} than partons due to momentum conservation (Becher+Neubert). But the 2nd line should still be present for bbH, since we have exactly 4 partons and we are working in $\overline{\text{MS}}$ scheme. What's going on?}.

Overall, despite the intimidating appearance of the formulas above, it can be appreciated that the only non-trivial piece of deriving the pole structure of the two-loop amplitudes is due to the operators $\bm{T}$, which we will refer to as `colour insertion operators'. Indeed, each $\bm{T}_i^a$ acts on the colour structure of the amplitude by inserting a gluon with the adjoint index $a$ onto the parton $i$. 

It might be very helpful to use pictures akin to Feynman diagrams in order to understand their action:


\subsubsection{Application to $\ppbbh$ amplitudes}
\texttt{Mathematica} package \texttt{ColorMath}~\cite{Sjodahl:2012nk}
\end{document}