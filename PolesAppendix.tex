\documentclass[main.tex]{subfiles}

\begin{document}
\renewcommand{\chaptername}{}
\renewcommand{\thechapter}{}
\renewcommand{\thetable}{C.\arabic{table}}
\renewcommand{\thesection}{C.\arabic{section}}
\renewcommand{\thesubsection}{C.\arabic{subsection}}
\renewcommand{\theequation}{C.\arabic{section}.\arabic{equation}}
\chapter{Appendix C} \label{app:polestructure}
In the following, we provide a detailed derivation of the full pole structure of the $\ppbbh$ amplitudes presented in Eqs.~\ref{eq:poles1L} to \ref{eq:H2bbggH}. We do this for completeness, but also in the hope of illuminating the procedure, which can be far from obvious. 
\section{UV singularities}
Let us start with the UV singularities. Renormalising the amplitude amounts to replacing the bare couplings in the Lagrangian with the physical ones according to:
\begin{equation}
    x_B = Z_x x_R \,.
\end{equation}
The renormalisation factors are expanded as $Z_x = 1 + \delta_x$, where $\delta_x$ are the counterterms added to the bare Lagrangian and which are designed precisely to cancel out the divergences. Thus, we can think of renormalisation as absorbing the infinities from $x_B$ into $Z_x$. Then, working with the renormalised parameters $x_R$ in the Lagrangian is known as `renormalised perturbation theory'. Let us now see how to achieve that. Specifying to the case of the $\ppbbh$ process in Sec.\ref{sec:Hbb}, the bare amplitude admits the following expansion (similar to that of Eq.~\ref{eq:ampexpansion})\footnote{See Refs.~\cite{Ahmed:2014pka, Mondini:2019vub} for details, but note the difference of $\alpha_{s,B}^{1/2}$ as the leading power due to one fewer jet and, in the second reference, the expansion in powers of $\alpha_{s,B}/(2 \pi)$, which introduces relative factors of $2$ with respect to our convention}:
\begin{equation} \label{eq:ampexpbare}
    A_B = a_B y_{b,B} \left(A_B^{(0)} + a_B A_B^{(1)} + a_B^2 A_B^{(2)} + \ldots \right)\,,
\end{equation}
where $a_B = \alpha_{s,B}/(4\pi)$ and $\alpha_{s,B}\,$,  $y_{b,B}$ are the bare strong coupling constant and the bare Yukawa coupling of the $b$ quark\JK{What about the renormalisation of $m_B$ and the wavefunction?}. We will work in the $\overline{\text{MS}}$ scheme and make the following replacements for the two couplings:
\begin{align}
    \alpha_{s,B} &= S_\eps\, Z_{\alpha_s} \, \alpha_{s,R}(\mu_R) \label{eq:baretorenormalpha} \\
    y_{b,B} &= Z_y \, y_{b,R}(\mu_R) \label{eq:baretorenormyb} \,,
\end{align}
where $\mu_R$ is the renormalisation scale. The factor $S_\eps = e^{\eps \gamma_E} (4\pi)^{-\eps}$ cancels out with the $m_\eps$ factor extracted from the colour-ordered amplitudes in Eq.~\ref{eq:colourdecomposition} and so will be dropped in the following discussion. The renormalisation factors themselves admit a perturbative expansion in the renormalised strong coupling constant:
\begin{align}
    Z_{\alpha_s} &= 1 + a_R r_1 + a_R^2 r_2 + \order{a_R^3}  \label{eq:Zalphacoeff} \\
    Z_y &= 1 + a_R s_1 + a_R^2 s_2 + \order{a_R^3} \label{eq:Zybcoeff}\,,
\end{align}
where $a_R = \alpha_{s,R}/(4\pi)$. Substituting Eqs.~\ref{eq:baretorenormalpha} through \ref{eq:Zybcoeff} into Eq.~\ref{eq:ampexpbare}, we obtain the amplitude expansion in terms of the renormalised parameters:
\begin{equation} \label{eq:ampexprenorm}
    A_R = a_R y_{b,R} \left(A_R^{(0)} + a_R A_R^{(1)} + a_B^2 A_R^{(2)} + \ldots \right)\,,
\end{equation}
with the individual amplitudes given by:
\begin{align}
    A_R^{(0)} &= A_B^{(0)} \,,\nonumber \\ 
    A_R^{(1)} &= A_B^{(1)} + 2 A_B^{(0)}(r_1 + s_1) \,, \label{eq:amprenorminbare}\\ 
    A_R^{(2)} &= A_B^{(2)} + 2 A_B^{(1)}(2r_1 + s_1) + 4A_B^{(0)} (r_2 + r_1 s_1 + s_2)\,. \nonumber
\end{align}
The coefficients $r_1, r_2, s_1, s_2$ are listed in Appendix~\ref{app:renormconstants}. Note that we use their values as given in Ref.~\cite{Mondini:2019vub} for the sake of consistency, but in Eq.~\ref{eq:amprenorminbare} we have rescaled them by powers of 2 to make them compatible with our expansion). Overall, we see that each UV-finite amplitude $A_R^{(L)}$ is defined by adding to the bare amplitude $A_B^{(L)}$ terms related to the $\beta$-function whose $\eps$ poles cancel out the UV divergences. This completes the renormalisation of the $\ppbbh$ amplitudes in Chapter~\ref{sec:Hbb}.
\section{IR singularities}
The renormalised amplitudes $A_R^{(L)}$ are only UV-finite --- the IR divergences remain. As explained in Sec.~\ref{sec:IRdivergencesandKLN}, the IR pole structure of two-loop amplitudes in massless gauge theories was originally derived in Ref.~\cite{Catani:1998bh} and later extended in  Refs.~\cite{Becher:2009cu, Becher:2009qa, Gardi:2009qi}. Here, we provide a brief summary of these results, followed by their application to our process of interest, $\ppbbh$.
\subsection{Overview of IR singularities}
The $\overline{\text{MS}}$-renormalised amplitudes $A_R^{(L)}$ admit the following structure\footnote{Note that similarly to the previous section, we rescale $\bm{I}^{(1)}(\eps)$ and $\bm{I}^{(2)}(\eps)$ by $2$ and $2^2$, respectively, to account for the different expansion parameter $a_R$ in Ref.~\cite{Catani:1998bh}.}:
\begin{align}
    A_R^{(1)} &= 2\bm{I}^{(1)}(\eps) A_R^{(0)} + F^{(1)}\,, \label{eq:A1IRstructure} \\
    A_R^{(2)} &= 2\bm{I}^{(1)}(\eps) A_R^{(1)} + 4\bm{I}^{(2)}(\eps) A_R^{(0)} + F^{(2)} \,. \label{eq:A2IRstructure}
\end{align}
Thus, the IR-divergent part is determined by amplitudes at lower loop order, while the full amplitude also receives a finite part $F^{(L)}$, which is the genuinely new contribution that needs to be computed. The pole operators are given by:
\begin{align}
    \bm{I}^{(1)}(\eps) &= \frac{1}{2} \frac{e^{\eps \gamma_E}}{\Gamma(1-\eps)} \sum_i \left(\frac{1}{\eps^2} - \frac{\gamma_0^i}{2\eps} \frac{1}{\bm{T}_i^2}\right) \sum_{j\neq i} \bm{T}_j \cdot \bm{T}_j \left(\frac{\mu_R^2}{-s_{ij}} \right)^\eps\,, \\
    \bm{I}^{(2)}(\eps) &= \frac{e^{-\eps \gamma_E} \Gamma(1-2\eps)}{\Gamma(1-\eps)} \left( \frac{\gamma_1^\text{cusp}}{8} + \frac{\beta_0}{2\eps} \right) \bm{I}^{(1)}(2\eps) - \frac{1}{2}\bm{I}^{(1)}(\eps) \left(\bm{I}^{(1)}(2\eps) + \frac{\beta_0}{\eps} \right) + \bm{H}_{\text{RS}}^{(2)}(\eps)\,,
\end{align}
where $s_{ij} = 2\sigma_{ij} p_i \cdot p_j$ with $\sigma_{ij}=+1$ if both momenta are incoming or outgoing and $\sigma_{ij}=-1$ otherwise. The values of the $\beta$-function coefficients and the anomalous dimensions are given in Appendix~\ref{app:renormconstants}. We remark that the one-loop pole operator $\bm{I}^{(1)}(\eps)$ is independent of the regularisation scheme, while the two-loop operator $\bm{I}^{(2)}(\eps)$ is not. Its dependence enters through the quantity $\bm{H}_{\text{RS}}^{(2)}(\eps)$, which contains $\order{1/\eps}$ poles only:
\begin{align} \label{eq:IR_H_function}
    \bm{H}_{\text{RS}}^{(2)}(\eps) &= \frac{1}{16\eps} \sum_i \left(\gamma_1^i - \frac{1}{4} \gamma_1^{\text{cusp}}\gamma_0^i + \frac{\pi^2}{16} \beta_0 \gamma_0^{\text{cusp}} C_i \right) \nonumber \\
    &+ \frac{i f^{abc}}{24\eps} \sum_{(i,j,k)} \bm{T}_i^a \bm{T}_j^b \bm{T}_k^c \ln\left(\frac{-s_{ij}}{-s_{jk}} \right) \ln\left(\frac{-s_{jk}}{-s_{kj}} \right) \ln\left(\frac{-s_{ki}}{-s_{ij}} \right) \\ 
    &-\frac{i f^{abc}}{128\eps} \gamma_0^{\text{cusp}} \sum_{(i,j,k)} \bm{T}_i^a \bm{T}_j^b \bm{T}_k^c \left( \frac{\gamma_0^i}{C_i} - \frac{\gamma_0^j}{C_j} \right)\ln\left(\frac{-s_{ij}}{-s_{jk}} \right) \ln\left(\frac{-s_{ki}}{-s_{ij}} \right)\,, \nonumber
\end{align}
where the sum runs over unordered tuples $(i,j,k)$ of distinct parton indices. As pointed out in Ref.~\cite{Becher:2009cu}, the last two lines appear only because the pole operators $\bm{I}^{(L)}(\eps)$ formulated in Ref.~\cite{Catani:1998bh} were not defined in a minimal scheme, but also include terms finite in $\eps$. Due to colour conservation, they contribute only if the amplitude contains at least four partons \JK{The last line appears only for \textit{more} than partons due to momentum conservation (Becher+Neubert). But the 2nd line should still be present for bbH, since we have exactly 4 partons and we are working in $\overline{\text{MS}}$ scheme. What's going on?}.

Overall, despite the intimidating appearance of the formulas above, it can be appreciated that the only non-trivial piece of deriving the pole structure of the two-loop amplitudes is due to the operators $\bm{T}_i^a$, which we will refer to as `colour insertion operators'. Indeed, each $\bm{T}_i^a$ acts on the colour structure of the amplitude by inserting a gluon with the adjoint index $a$ onto the parton $i$. The rules are as follows:
\begin{itemize}
    \item $\bm{T}_c^a: \delta^{bc} = -\ii f^{abc}$ if $c$ is a gluon
    \item $\bm{T}_i^a: \delta_{ij} = + (T^a)_{ij}$ if $i$ is an outgoing quark 
    \item $\bm{T}_i^a: \delta_{ji} = - (T^a)_{ji}$ if $i$ is an outgoing antiquark
    \item $\bm{T}_i^a: \delta_{ji} = - (T^a)_{ji}$ if $i$ is an incoming quark
    \item $\bm{T}_i^a: \delta_{ij} = + (T^a)_{ij}$ if $i$ is an incoming antiquark
\end{itemize}
We find it very helpful to use pictures akin to Feynman diagrams in order to better understand these rules (see  Fig.~\ref{fig:colourinsertion}). 
\begin{figure}
    \centering
    %\begin{adjustbox}{minipage=\textwidth,scale=0.9}
    %\raisebox{0.2cm}{
    \begin{subfigure}[b]{0.8\linewidth}
        \hspace*{1em}
        \begin{tikzpicture}[every text node part/.style={align=left}]
    	\begin{feynman}[small]
    		\vertex (v) node[anchor=north] {$b$};
            \vertex (f) [right=2cm of v] {$\phantom{i_1}$};
            \node at (f.south west) {$c$};
            \node (t) [left=0.5cm of v] {$\bm{T}^a_c:$};
            \node (eq) [right=0.5cm of f] {$=$};

            \vertex (v2) [right=0.5cm of eq] {$-\ii$};
            \node at (v2.south east) {$b$};
            \vertex (m) [right=1.3cm of v2] [anchor=west];
            \vertex (f2) [right=1.2cm of m] {\phantom{j}};
            \node at (f2.south west) {$c$};
            \vertex (g) [above=of m] {$a$};

            \node (text) [right=1.7cm of f2] {gluon};
            
    		\diagram*{
    			(v) -- [thick, gluon] (f);
                (v2) -- [thick, gluon] (m) -- [thick, gluon] (f2);
                (m) -- [thick, gluon] (g);
    		};
    	\end{feynman}
    \end{tikzpicture}
    \end{subfigure}
    \begin{subfigure}[b]{0.8\linewidth}
        \hspace*{1em}
        \begin{tikzpicture}[every text node part/.style={align=left}]
    	\begin{feynman}[small]
    		\vertex (v) node[anchor=north] {$j$};
            \vertex (f) [right=2cm of v] {$\phantom{i_1}$};
            \node at (f.south west) {$i$};
            \node (t) [left=0.5cm of v] {$\bm{T}^a_i:$};
            \node (eq) [right=0.5cm of f] {$=$};

            \vertex (v2) [right=0.5cm of eq] {$+$};
            \node at (v2.south east) {$j$};
            \vertex (m) [right=1.3cm of v2] [anchor=west];
            \vertex (f2) [right=1.2cm of m] {\phantom{j}};
            \node at (f2.south west) {$i$};
            \vertex (g) [above=of m] {$a$};

            \node (text) [right=2cm of f2] {outgoing\\quark};
            
            \filldraw (v) circle (2pt);
            \filldraw (v2.east) circle (2pt);
            
    		\diagram*{
    			(v) -- [thick, fermion] (f);
                (v2) -- [thick, fermion] (m) -- [thick, fermion] (f2);
                (m) -- [thick, gluon] (g);
    		};
    	\end{feynman}
    \end{tikzpicture}
    \end{subfigure}
    \begin{subfigure}[b]{0.8\linewidth}
        \hspace*{1em}
        \begin{tikzpicture}[every text node part/.style={align=left}]
    	\begin{feynman}[small]
    		\vertex (v) node[anchor=north] {$j$};
            \vertex (f) [right=2cm of v] {$\phantom{i_1}$};
            \node at (f.south west) {$i$};
            \node (t) [left=0.5cm of v] {$\bm{T}^a_i:$};
            \node (eq) [right=0.5cm of f] {$=$};

            \vertex (v2) [right=0.5cm of eq] {$-$};
            \node (c) at (v2.south east) {$j$};
            \vertex (m) [right=1.3cm of v2] [anchor=west];
            \vertex (f2) [right=1.2cm of m] {\phantom{j}};
            \node at (f2.south west) {$i$};
            \vertex (g) [above=of m] {$a$};

            \node (text) [right=2cm of f2] {outgoing\\antiquark};
            
            \filldraw (v) circle (2pt);
            \filldraw (v2.east) circle (2pt);
            
    		\diagram*{
    			(v) -- [thick, anti fermion] (f);
                (v2) -- [thick, anti fermion] (m) -- [thick, anti fermion] (f2);
                (m) -- [thick, gluon] (g);
    		};
    	\end{feynman}
    \end{tikzpicture}
    \end{subfigure}
    \begin{subfigure}[b]{0.8\linewidth}
        \hspace*{1em}
        \begin{tikzpicture}[every text node part/.style={align=left}]
    	\begin{feynman}[small]
    		\vertex (v) node[anchor=north] {$i$};
            \vertex (f) [right=2cm of v] {$\phantom{i_1}$};
            \node at (f.south west) {$j$};
            \node (t) [left=0.5cm of v] {$\bm{T}^a_i:$};
            \node (eq) [right=0.5cm of f] {$=$};

            \vertex (v2) [right=0.5cm of eq] {$-$};
            \node at (v2.south east) {$i$};
            \vertex (m) [right=1.3cm of v2] [anchor=west];
            \vertex (f2) [right=1.2cm of m] {\phantom{j}};
            \node at (f2.south west) {$j$};
            \vertex (g) [above=of m] {$a$};

            \node (text) [right=2cm of f2] {incoming\\quark};
            
            \filldraw (f.west) circle (2pt);
            \filldraw (f2.west) circle (2pt);
            
    		\diagram*{
    			(v) -- [thick, fermion] (f);
                (v2) -- [thick, fermion] (m) -- [thick, fermion] (f2);
                (m) -- [thick, gluon] (g);
    		};
    	\end{feynman}
    \end{tikzpicture}
    \end{subfigure}
    \begin{subfigure}[b]{0.8\linewidth}
        \hspace*{1em}
        \begin{tikzpicture}[every text node part/.style={align=left}]
    	\begin{feynman}[small]
    		\vertex (v) node[anchor=north] {$i$};
            \vertex (f) [right=2cm of v] {$\phantom{i_1}$};
            \node at (f.south west) {$j$};
            \node (t) [left=0.5cm of v] {$\bm{T}^a_i:$};
            \node (eq) [right=0.5cm of f] {$=$};

            \vertex (v2) [right=0.5cm of eq] {$+$};
            \node at (v2.south east) {$i$};
            \vertex (m) [right=1.3cm of v2] [anchor=west];
            \vertex (f2) [right=1.2cm of m] {\phantom{j}};
            \node at (f2.south west) {$j$};
            \vertex (g) [above=of m] {$a$};

            \node (text) [right=2cm of f2] {incoming\\antiquark};
            
            \filldraw (f.west) circle (2pt);
            \filldraw (f2.west) circle (2pt);
            
    		\diagram*{
    			(v) -- [thick, anti fermion] (f);
                (v2) -- [thick, anti fermion] (m) -- [thick, anti fermion] (f2);
                (m) -- [thick, gluon] (g);
    		};
    	\end{feynman}
    \end{tikzpicture}
    \end{subfigure}
    %}
    %\end{adjustbox}
\caption{Graphical representation of the action of the colour insertion operators on partons. The dot $\bullet$ indicates a Feynman diagram vertex and allows us to distinguish between incoming and outgoing quarks and antiquarks. The fundamental and antifundamental indices should be read in the direction opposite to fermion flow, while the adjoint indices in the three-point gluon vertex should be read anticlockwise.\JK{Does that make sense?}}
\label{fig:colourinsertion}
\end{figure}
Before applying them to our process of interest, it is useful to note that the product of two colour insertion operators, $\bm{T}_i\cdot\bm{T}_j \equiv \bm{T}_i^a \bm{T}_j^a$, trivially commutes: 
\begin{equation} \label{eq:TiTjcommute}
    \bm{T}_i\cdot\bm{T}_j = \bm{T}_j\cdot\bm{T}_i\,,
\end{equation}
while for $i=j$ the action of the product gives the quadratic Casimir of the appropriate representation of $SU(N_c)$:
\begin{equation} \label{eq:Ti^2eqCi}
    \bm{T}_i^2 = C_i\,,
\end{equation}
with $C_q = C_{\bar{q}} = C_F$ and $C_g=C_A$. Moreover, due to colour conservation:
\begin{align} 
    \sum_{i=1}^n \bm{T}_i^a \ket{A_n} &= 0\,, \label{eq:poleinsertioncheckA} \\
    \sum_{(i,j)} \bm{T}_i \cdot \bm{T}_j = -\sum_i \bm{T}_i^2 &= -\sum_i C_i\,, \label{eq:poleinsertioncheckB}
\end{align}
where $\ket{A_n}$ denotes a vector in the $n$-dimensional colour space, with $A_n$ being the UV-renormalised amplitude of $n$ coloured partons\footnote{For a description of the colour-space formalism, see Ref.~\cite{Catani:1996vz}.}. These equations can be used to check whether we have applied the colour insertion operators to our amplitude correctly.
\subsection{Application to $\ppbbh$ amplitudes}
We now have all the tools to derive the IR pole structure of the $\ppbbh$ amplitudes in Chapter~\ref{sec:Hbb}. First of all, we note that by applying the first line of Eq.~\ref{eq:IR_H_function} to the $\bbqqh$ and $\bbggh$ channels, Eqs.~\ref{eq:H2bbqqH} and \ref{eq:H2bbggH} follow trivially. Then, to derive the two-loop pole operator $\bm{I}^{(2)}(\eps)$, we only need to concern ourselves with the action of the colour insertion operators $\bm{T}_i^a$ within $\bm{I}^{(1)}(\eps)$.
\subsubsection{The $\bbqqh$ channel}
As the first step, we write down all the colour structures that appear in the relevant Feynman diagrams. Note that factors such as $N_c, T_F, C_A$ or $C_F$ are not considered a part of these structures --- we only include $\delta$-functions, fundamental generators $(T^a)_{ij}$ or the structure constants $f^{abc}$\JK{Maybe this sentence can be improved.}. For this channel, we find two structures (see Fig.~\ref{fig:bbqqHcolour}):
\begin{equation}
    \bm{c} = 
    \begin{pmatrix}
        \delta_{i_4}^{\;\;\bar{i}_1} \delta_{i_2}^{\;\;\bar{i}_3} & \delta_{i_2}^{\;\;\bar{i}_1} \delta_{i_4}^{\;\;\bar{i}_3}
    \end{pmatrix}^T\,.
\end{equation}
Then, the action of the products $\bm{T}_i \cdot \bm{T}_j$ produces a linear combination of these structures, which we encode in the matrices $\mathcal{C}_{ij}$ defined as follows:
\begin{equation}
    \bm{T}_i \cdot \bm{T}_j :  \bm{c} = \mathcal{C}_{ij} \bm{c}\,.
\end{equation}
Note that we do not have to compute $\mathcal{C}_{ij}$ for all possible combinations of $i$ and $j$. Due to Eq.~\ref{eq:Ti^2eqCi}, all $\mathcal{C}_{ii}$ are diagonal and their entries are given by the quadratic Casimirs $C_i$ in the relevant representation. Moreover, the commutativity property Eq.~\ref{eq:TiTjcommute} further reduces the number of necessary computations. Overall, for $n$ coloured partons, we only need to compute $n(n+1)/2$ matrices.
\begin{figure}
    \centering
        \begin{tikzpicture}
        	\begin{feynman}[small]
        		\vertex (v1);
          
        		\vertex[above left = 1cm of v1] (i1) {$i_4$};
        		\vertex[below left =  1cm of v1] (i2) {$\bar{i}_3$};
        		\vertex[right = of v1] (v2);			
        		
        		\vertex[above right = 1cm of v2] (f1) {$\bar{i}_1$};
        		\vertex[below right = 1cm of v2] (f2) {$i_2$};

                %2nd diagram
                \vertex[below right=of f1, xshift=1.5cm] (2i1) {$i_4$};
                \vertex[above right=of f2, xshift=1.5cm] (2i2) {$\bar{i}_3$};
                		
        		\vertex[right=2cm of 2i1] (2f1) {$\bar{i}_1$};
                \vertex[right=2cm of 2i2] (2f2) {$i_2$}; 

                %3rd diagram
                \vertex[above right=of 2f1, xshift=1cm] (3i1) {$i_4$};
                \vertex[below right=of 2f2, xshift=1cm] (3i2) {$\bar{i}_3$};
                		
        		\vertex[right=of 3i1] (3f1) {$\bar{i}_1$};
                \vertex[right=of 3i2] (3f2) {$i_2$}; 

                %equation nodes
                \node[right=2.5cm of v1] {$=T_F$ \Huge $\Biggr($};
                \node[right=6.5cm of v1] {$ -\,\,\dfrac{1}{N_c} $};
                \node[right=9cm of v1] {\Huge $\Biggr)$};

        		\diagram*{
                    %1st diagram
        			(i1) -- [anti fermion] (v1) -- [anti fermion] (i2),
                    (v1) -- [gluon] (v2),
        			(f1) -- [fermion] (v2) -- [fermion] (f2),
                    %2nd diagram
                    (2i1) -- [anti fermion] (2f1),
                    (2i2) -- [fermion] (2f2),
                    %3rd diagram
                    (3i1) -- [anti fermion] (3i2),
                    (3f1) -- [fermion] (3f2),
        		};
        	\end{feynman}
        \end{tikzpicture}
    \caption{The colour structure of a sample diagram contributing to the $\bbqqh$ channel at tree level. We employ the Fierz identity to write $(T^a)_{i_4}^{\;\;\bar{i}_3} (T^a)_{i_2}^{\;\;\bar{i}_1} = T_F \left( \delta_{i_4}^{\;\;\bar{i}_1} \delta_{i_2}^{\;\;\bar{i}_3} - \frac{1}{N_c} \delta_{i_2}^{\;\;\bar{i}_1} \delta_{i_4}^{\;\;\bar{i}_3}\right)$. Note that particles 3 and 4 have to be crossed into the final state to correspond to the process definition in Eq.~\ref{eq:subprocessqq}, where all particles are outgoing. Thus, we should treat them as an outgoing antiquark and an outgoing quark, respectively, when applying the rules from Fig.~\ref{fig:colourinsertion}.} 
    \label{fig:bbqqHcolour}
\end{figure}

As an example, let us see how the operator product $\bm{T}_1 \cdot \bm{T}_4$ acts on the two colour structures in this channel. Applying the rules listed in Fig.~\ref{fig:colourinsertion} (remember that particle 4 is treated as an outgoing quark):
\begin{align}
    \bm{T}_1 \cdot \bm{T}_4 : \delta_{i_4}^{\;\;\bar{i}_1} \delta_{i_2}^{\;\;\bar{i}_3} &= -(T^a T^a)_{i_4}^{\;\;\bar{i}_1} \delta_{i_2}^{\;\;\bar{i}_3} = -C_F \delta_{i_4}^{\;\;\bar{i}_1} \delta_{i_2}^{\;\;\bar{i}_3}\,, \\
    \bm{T}_1 \cdot \bm{T}_4 : \delta_{i_2}^{\;\;\bar{i}_1} \delta_{i_4}^{\;\;\bar{i}_3} &= -(T^a)_{i_2}^{\;\;\bar{i}_1} (T^a)_{i_4}^{\;\;\bar{i}_3} = -T_F \left(\delta_{i_4}^{\;\;\bar{i}_1}\delta_{i_2}^{\;\;\bar{i}_3} - \frac{1}{N_c}\delta_{i_2}^{\;\;\bar{i}_1} \delta_{i_4}^{\;\;\bar{i}_3} \right) \,,
\end{align}
where in both lines we have used the Fierz identity in the last equality. Perhaps a graphical representation of these operations in Fig.~\ref{fig:qqinsertion} is once again more illustrative. We can now read off the matrix $\mathcal{C}_{14}$:
\begin{equation}
    \mathcal{C}_{14} = 
    \begin{pmatrix}
    -C_F & 0 \\
    -T_F & \frac{T_F}{N_c}
\end{pmatrix}\,.
\end{equation}
In the same manner, we need to obtain the remaining matrices $\mathcal{C}_{ij}$. Needless to say, performing these operations by hand (or even using their diagrammatical equivalents) becomes tedious and extremely prone to errors. We therefore automate this task in \texttt{Mathematica} and make use of the package \texttt{ColorMath}~\cite{Sjodahl:2012nk} to achieve the simplifications of the various colour structures that arise as a result of applying the colour insertion operators. We also verify that Eq.~\ref{eq:poleinsertioncheckB} holds as a check on our calculations.

Once all the matrices $\mathcal{C}_{ij}$ have been computed, we have all the information needed to act with the pole operators $\bm{I}^{(L)}$ on the renormalised amplitudes $A_R^{(L)}$ and subtract the IR singularities. Note that if we want to retain full colour dependence, for each colour-ordered amplitude the subtraction will involve multiple lower-loop amplitudes with different colour factors. However, within the leading-colour approximation, the situation becomes much simpler. Indeed, replacing each $\bm{T}_i \cdot \bm{T}_j$ with the corresponding $\mathcal{C}_{ij}$ (and each $\bm{T}_i^2$ with the corresponding quadratic Casimir $C_i$), we retain only the highest power of $N_c$ to find:
\begin{equation}
    \bm{I}_{\bbqqh}^{(1)} = -N_c \frac{N(\eps)}{2}\left(\frac{1}{\eps^2} + \frac{3}{2\eps} \right) \left((-s_{23})^{-\eps} + (-s_{14})^{-\eps}\right)
    \begin{pmatrix}
        1 & 0 \\
        0 & 1
    \end{pmatrix}\,.
\end{equation}
Moreover, it turns out that the second colour factor $\delta_{i_2}^{\;\;\bar{i}_1} \delta_{i_4}^{\;\;\bar{i}_3}$ always appears suppressed by $1/N_c$ with respect to the first factor $\delta_{i_4}^{\;\;\bar{i}_1} \delta_{i_2}^{\;\;\bar{i}_3}$, as can be appreciated from our simple example in Fig.~\ref{fig:bbqqHcolour}. Thus, in the leading-colour approximation, we simply drop it. This justifies the decomposition in Eq.~\ref{eq:colourdecomposition} and completes the derivation of the pole operator $I_{\bbqqh}^{(1)}$ in Eq.~\ref{eq:I1bbqqh}, which now acts solely on the amplitude associated with the colour factor $\delta_{i_4}^{\;\;\bar{i}_1} \delta_{i_2}^{\;\;\bar{i}_3}$.

Finally, we combine the UV and IR poles in a single operator $P^{(L)}$ which we can use to subtract both types of divergences from the amplitude. Substituting Eq.~\ref{eq:amprenorminbare} into Eqs.~\ref{eq:A1IRstructure} and \ref{eq:A2IRstructure}, we obtain the finite part:
\begin{align}
    F^{(1)} &= A_R^{(1)} - 2 I^{(1)}(\eps) A_R^{(0)} \nonumber \\
            &= A_B^{(1)} + 2 A_B^{(0)}(r_1 + s_1) - 2 I^{(1)}(\eps) A_B^{(0)} \nonumber \\
            &= A_B^{(1)} - \underbrace{2 \left(I^{(1)}(\eps) - r_1 - s_1 \right)}_{P^{(1)}} A_B^{(0)} \,, \\
            &\nonumber \\
    F^{(2)} &= A_R^{(2)} - 2 I^{(1)}(\eps) A_R^{(1)} - 4 I^{(2)}(\eps) A_R^{(0)} \nonumber \\
            &= A_B^{(2)} + 2 A_B^{(1)}(2r_1 + s_1) + 4A_B^{(0)} (r_2 + r_1 s_1 + s_2) \nonumber \\
            & -2 I^{(1)}(\eps) \left(A_B^{(1)} + 2 A_B^{(0)}(r_1 + s_1)\right) - 4 I^{(2)}(\eps) A_B^{(0)} \nonumber \\
            &= A_B^{(2)} - \underbrace{\left(2 I^{(1)}(\eps) \left(\hat{A}_B^{(1)} + r_1 + s_1\right) + 4I^{(2)}(\eps) - 2 (2r_1 + s_1) \hat{A}_B^{(1)} -  4(r_2 + r_1 s_1 + s_2) \right)}_{P^{(2)}} A_B^{(0)}\,.
\end{align}
\subsubsection{The $\bbggh$ channel}
\end{document}