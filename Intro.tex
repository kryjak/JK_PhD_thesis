\documentclass[main.tex]{subfiles}
\begin{document}
\chapter{Introduction} \label{sec:intro}
\begin{figure}[h]
    \centering
    \begin{tikzpicture}
        \randuck
    \end{tikzpicture}
    \caption{A duck with a label after caption.}
    \label{fig:after}
    This is referencing Fig.~\ref{fig:after}.
\end{figure}
\section{QED} \label{sec:QEDintro}
The central equation of quantum field theory which describes all spin\=/1/2 particles is the Dirac equation:
\begin{equation} \label{eq:Dirac}
    (\slashed{p}-m)\psi(x) = 0\,.
\end{equation}
The solutions to this equation are given by the four-component Dirac spinors $\psi(x)$, which transform in the irreducible $\left(\frac{1}{2}, 0\right) \otimes \left(0, \frac{1}{2}\right)$ representation\footnote{Not to be confused with the usual four-vectors $x^\mu$, which transform in the $\left(\frac{1}{2}, \frac{1}{2}\right)$ representation.}. There are four plane-wave solutions to the Dirac equation, which admit both positive and negative energies:
\begin{equation}
    \psi(x) = u^s(p)\ex{-\i p\cdot x} \qquad \text{and} \qquad \psi(x) = v^s(p)\ex{+\i p\cdot x}\,,
\end{equation}
where $s=\{+, -\}$ labels the spin of the fermion. The momentum space spinors $u(p)$ and $v(p)$ are interpreted as describing particles and antiparticles with \textit{positive} energies, respectively. They satisfy the momentum space Dirac equations: 
\begin{align}
    (\slashed{p} - m)u(p) &= 0\,, \\
    (\slashed{p} + m)v(p) &= 0\,,
\end{align}
as well as the following completeness relations, which is useful when performing spin sums in amplitude computations:
\begin{align} \label{eq:completeness:fermions}
    \sum_{s=\pm} u^s(p)\bar{u}^s(p) &= \slashed{p} + m\,, \\
    \sum_{s=\pm} v^s(p)\bar{v}^s(p) &= \slashed{p} - m\,.
\end{align}
Moreover, it is convenient to split the spinors into two separate components:
\begin{equation}
    \psi(x) = 
    \begin{pmatrix}
        \psi_L(x) \\
        \psi_R(x)
    \end{pmatrix}\,,
\end{equation}
where the left- and right-handed Weyl spinors transform under the $\left(\frac{1}{2}, 0 \right)$ and $\left(\frac{1}{2}, 0 \right)$ representations, respectively. Crucially, in the absence of mass, the two components fully decouple \JK{maybe a better word?}. It is also useful to define an additional $\gamma^5$ matrix (also known as the chirality operator): 
\begin{equation}
    \gamma^5 \equiv \i \gamma^0 \gamma^1 \gamma^2 \gamma^3\,,
\end{equation}
which in the Weyl (chiral) representation of the $\gamma$-matrices is just:
\begin{equation}
    \gamma^5 = 
    \begin{pmatrix}
        1 & 0 \\
        0 & -1
    \end{pmatrix}\,.
\end{equation}
Then, the Weyl spinors can be extracted from the Dirac spinors by applying appropriate projectors, according to $\psi_L(x) = P_L\psi(x)$ and $\psi_R(x) = P_R\psi(x)$:
\begin{equation}
    P_L = \frac{1-\gamma^5}{2} = 
    \begin{pmatrix}
        0 & 0 \\
        0 & 1
    \end{pmatrix}\,,
    \qquad
    P_R = \frac{1+\gamma^5}{2} = 
    \begin{pmatrix}
        1 & 0 \\
        0 & 0
    \end{pmatrix}\,.
\end{equation}
Finally, we can associate $\psi_L$ and $\psi_R$ with the \textbf{helicity} of a particle, which is the projection of its spin onto the direction of motion. The corresponding helicity values for left- and right-handed particles are $-1$ and $+1$, respectively\footnote{Strictly speaking, the left- and right-handed Weyl spinors are eigenstates of the chirality operator $\gamma^5$. However, for $m=0$, helicity and chirality eigenstates coincide}.
Weyl spinors will play a key role in the discussion of the spinor-helicity formalism in Section~\ref{sec:spinhelform}.

\section{Quantum Chromodynamics} \label{sec:QCD}
\begin{equation} \label{eq:liealgebra}
    [T^{a_1}, T^{a_2}] = \i f^{a_1a_2a_3} T^{a_3} \,.
\end{equation}
Working in the fundamental representation with normalisation $\tr(T^{a_1} T^{a_2}) = \mathrm{T}_F \, \delta^{a_1a_2}$, we multiply both sides by $T^{a_3}$ and take the trace:
\begin{align} \label{eq:f(abc)}
    \tr(T^{a_3}[T^{a_1}, T^{a_2}]) &= \i f^{a_1a_2a_4} \tr(T^{a_3} T^{a_4}) \nonumber \\
    \tr(T^{a_3} T^{a_1} T^{a_2} - T^{a_3} T^{a_2} T^{a_1}) &= if^{a_1a_2a_4} \mathrm{T}_F \delta^{a_3a_4} \nonumber \\ 
    \implies f^{a_1a_2a_3} &= -\frac{\i}{\mathrm{T}_F} \tr(T^{a_3} T^{a_1} T^{a_2} - T^{a_3} T^{a_2} T^{a_1}) \nonumber \\
    &= -\frac{\i}{\mathrm{T}_F} \tr(T^{a_1} T^{a_2} T^{a_3} - T^{a_1}T^{a_3} T^{a_2}) \,.
\end{align}

Completeness relations for the spin-1 polarisation vectors:
\begin{equation} \label{eq:completeness:bosons}
    \sum_{s=\pm} \varepsilon_\mu^s(p, q) (\varepsilon_\nu^s(p, q))^{\ast} = -g_{\mu\nu} + \frac{p_\mu q_\nu + p_\nu q_\mu}{p \cdot q}
\end{equation}
\JK{is this true only in the axial gauge? See Eq. 30 of `Calculating scattering amplitudes efficiently'. We don't use the axial gauge.}

\begin{equation} \label{eq:QCDbetafunction}
    \text{QCD Beta function - asymptotic freedom}
\end{equation}

\section{Particle scattering} \label{sec:particlescattering}
Having introduced the quantum field theories that describe the most fundamental particles our nature has to offer, one might rightfully wonder about the rewards for our intellectual effort (beyond personal satisfaction). After all, amongst our non-physicist friends, it is common to think that the word `theoretical' in theoretical physics is synonymous with `hypothetical'. Not much could be further from the truth and we dedicate this section to illustrating the immense predictive power of QFTs.

Consider a typical particle physics experiment. Two beams of particles, travelling in opposite direction, interact briefly and produce a host of new states that fly away in various directions until they are registered by some kind of a detector. Naturally, the number of such scattering events is proportional to the number of particles in the two beams, $N_a$ and $N_b$, as well as to the area common to the beams, $A$. The ratio of these quantities is known as the scattering \textbf{cross-section}:
\begin{equation} \label{eq:crossection}
    \sigma \equiv \frac{\text{number of scattering events}}{N_a N_b A}\,.
\end{equation}
Typically, apart from just counting the total number of events, we want to differentiate between the type of outgoing particles, their momentum, angle of collision, etc. However, if we specify an exact value for a continuous variable such as the momentum, the numerator of Eq.~\ref{eq:crossection} becomes infinitesimal. To avoid this issue, we usually work with a \textit{differential} cross-section, e.g. $\dd\sigma/(\dd p_1 \ldots \dd p_n)$, such that its integral over some small range of $p_i$ gives the \textit{total} cross-section in that region of momentum space. While cross-sections can be measured in particle colliders, it is the task of QFT to make predictions for them.

In the case of hadron collisions, there is one additional complication which is not present in the relatively `clean' world of QED interactions\JK{What about the `partons' of QED? One can also introduce PDFs for virtual electrons + photons in collinear emission. See Peskin Ch 17.5.}. The fundamental objects of QCD are quark and gluons, however they cannot be observed individually due to colour confinement --- only colourless combinations can be scattered and subsequently detected in a collider. The LHC, being (primarily) a proton-proton collider, requires a more complicated model of the interactions than just the scattering of point particles. Now, recall from Eq.~\ref{eq:QCDbetafunction} that QCD exhibits asymptotic freedom. The coupling $\alpha_s$ decreases with energy (i.e. at small distances) and its value determines whether we are allowed to use perturbation theory or not. Roughly speaking, above the scale $\Lambda_\text{QCD} \sim 200 \text{ GeV}$, $\alpha_s$ becomes small enough so that the perturbative expansion is justified. The individual `partons' (i.e. quarks, antiquarks and gluons) act like free particles and undergo `hard' scattering\cite{PhysRevLett.23.1415}. On the other hand, physics at energies below $\Lambda_\text{QCD}$ is necessarily non-perturbative due to large $\alpha_s$ values. Its effects are captured by the Parton Distribution Functions (PDFs) $f_i(x, Q)$, which give the probability of finding in a hadron a given parton $i$ with momentum fraction $x$ of this hadron\JK{Is this sentence grammatically correct?.}. Here, $Q$ is the resolution scale at which we're probing the hard scattering process\JK{Should I replace $Q$ with $\mu_F$ and call it the factorisation scale?}. In fact, the PDFs run with $Q$, similarly to the running of the QCD coupling $\alpha_s$, and their evolution is given by the DGLAP equations.
%Scaling with $\mu_F$ - DGLAP equations.
%PDFs are renormalised
Overall, the factorisation of soft and hard physics allows us to write the cross-section for the scattering of two hadrons as:

\begin{equation} \label{eq:factorisation}
    \dd \sigma = \sum_{i,j} \int_0^1 \dd x_1 \dd x_2 f_i(x_1, Q) f_j(x_2, Q) \dd \hat{\sigma}_{ij} + \order{\frac{\Lambda_\text{QCD}}{Q}}\,,
\end{equation}
\JK{Should the $Q$ in the higher order term be different to the $Q$ in the PDFs?}where $i, j$ are the partons of the two hadrons and $\hat{\sigma}_{ij}$ is the \textit{partonic} cross-section for the scattering of $i,j$ into the final state. Since the PDFs describe effects at large $\alpha_s$, they cannot be calculated using perturbative QCD. Instead, they are extracted from experimental data (see e.g. \cite{H1:2015ubc, Alekhin:2017kpj, Hou:2019efy, NNPDF:2021uiq, Buckley:2014ana}. On a positive note, they are universal, i.e. process independent, so once they have been determined from some experiment, they can be re-used in the description of any other hadron interaction. 

Contrary to the PDFs, the partonic cross-sections are computed using perturbative QCD. Before turning our full attention to these objects, let us briefly remark that even after the scattering has taken place, a lot of interesting physics is still taking place. The final state partons can radiate a cascade of other partons in a phenomenon referred to as parton showers. Furthermore, these coloured partons cannot be observed in an experiment --- as the energy scale falls below $\Lambda_\text{QCD}$, they combine together to form colourless hadrons. Finally, unstable hadrons may decay into further products. What is actually detected in a collider is then a collimated spray of hadrons and other particles referred to as a jet. Using the so-called jet algorithms and definitions, one then tries to reverse engineer the collision of individual partons\JK{Is that a correct thing to say?}. We refer the reader to Refs.~\cite{Plehn:2009nd, Hoche:2014rga} for an introduction on these topics.
\JK{insert a picture showing factorisation and then parton showers etc.}
\section{From cross-sections to scattering amplitudes}
Having decoupled the high and low energy physics within a hadron, let us now address the computation of the hard scattering between partons. The partonic cross section $\hat{\sigma}_{i\rightarrow f}$ is simply related to the probability of starting with some initial state $i$ and ending up with the final state $f$, that is: $\mathcal{P} = |\braket{f|i}|^2$. These interacting states are not the free wave packets with well-defined momenta that we know how to handle using the QFT machinery. However, they do become free in the limit as $t\rightarrow -\infty$, i.e. before the interaction, as well as $t \rightarrow \infty$, i.e. after the interaction. The idea is then to view the scattering in the following way: an asymptotically free state $\ket{i}_{t=-\infty}$ evolves to a real-world state $\ket{i}$, undergoes scattering (of negligible duration) into the state $\ket{f}$ which finally evolves into the free state $\ket{f}_{t=\infty}$. The evolution of real states from and to the asymptotic states is captured by the $S$-matrix, which is calculated from the Hamiltonian in the interaction representation. Specifying to the scattering of two partons into $n-2$ particles, we write the so-called $S$-matrix elements in terms of free multi-particle states:
\begin{equation} \label{eq:Smatrixelements}
    \braket{p_3, \ldots, p_n|S|p_1,p_2}\,.
\end{equation}
Furthermore, it is customary to split the $S$-matrix according to:
\begin{equation} \label{eq:StoiT}
    S = \mathds{1} + \i T\,.
\end{equation}
In any scattering experiment, there is a large chance of the particles simply missing each other and nothing happening, which is expressed by the identity matrix. The interesting physics of interactions is captured by the transition matrix $T$:
\begin{equation} \label{eq:Tmatamplitudes}
    \braket{p_3, \ldots, p_n|\i T|p_1,p_2} = (2\pi)^4 \delta^{(4)}\left(p_1+p_2 - \sum_{f=3}^n p_f\right)\, \i \mathcal{A} (p_1,p_2 \rightarrow p_3, \ldots, p_n) \,,
\end{equation}
where we included the overall $\delta$-function to impose 4-momentum conservation. Here, $\mathcal{A} (p_1,p_2 \rightarrow p_3, \ldots, p_n)$ is the $n$-particle \textbf{scattering amplitude}. Finally, to relate the amplitudes to the partonic cross-section, we insert Eq.~\ref{eq:Tmatamplitudes} into Eq.~\ref{eq:Smatrixelements} and integrate over a small region $\dd^3p_3 \ldots \dd^3 p_n$. This integration encodes the probability that the $n-2$ final particles will belong to that region of the momentum phase-space. Overall, the recipe reads:
\begin{equation} \label{eq:crosssecampl}
    \dd \hat{\sigma} = \frac{1}{2E_1 2E_2 |v_1-v_2|} \left(\prod_f \frac{\dd^3 p_f}{(2\pi)^3} \frac{1}{2E_f} \right) (2\pi)^4 \delta^{(4)}\left(p_1+p_2 - \sum_f p_f\right)|\mathcal{A}(p_1,p_2 \rightarrow p_f)|^2\,.
\end{equation}
The first term in this formula is a flux factor related to the relative velocity of the two incoming beams in the laboratory reference frame, $|v_1-v_2|$. Next, the overall $\delta$-function imposes Lorentz invariance on the phase-space integration over $\prod_f \dd^3 p_f$. In practice, the phase-space integration is carried out using Monte Carlo event generators~\cite{Reuschle:2014fya, Campbell:2022qmc}. Finally, the formula is completed by the square of the scattering amplitude.

Let us stress that the computation of each element described so far, i.e. the cross-sections, PDFs, the phase-space integrations, amplitudes, parton showers, hadronisation and jets, is a tremendous effort undertaken by a multitude of particle physicists around the world. This thesis is a small building block of the entire enterprise and aims to shed light on just one of these aspects --- the calculation of scattering amplitudes\footnote{Naturally, if we are not dealing with the scattering of hadrons, one can drop the `hats' on the cross-sections and forget about the PDFs, hadronisation, etc. In this case, the cross-section $\sigma$ is the hard scattering cross-section from the start.\JK{Is this really true? I think there are PDFs for leptonic processes too.}}.
\section{Scattering amplitudes}
Scattering amplitude - bridge between theory an experiment, due to Eq.~\ref{eq:crosssecampl}. \\
Cannot be calculated exactly, perturbative expansion of the amplitudes in $\alpha_s$.\\
Means that $\dd \sigma$ also admits expansion in $\alpha_s \rightarrow LO, ..., N^kLO$.
Each term in the perturbative expansion is represented by Feynman diagrams/integrals.\\
Loop diagrams/integrals - divergences.\\
Dimensional regularisation\\
KLN theorem \\
The next chapter is dedicated to a detailed description of the tools used to compute scattering amplitudes.

\subsection{Kinematics} \label{sec:kinematics}
\begin{align} \label{eq:delta3}
    \Delta_3^{(1)} &= \lambda(s_5, s_{12}, s_{34}) \nonumber \\
    \Delta_3^{(2)} &= \lambda(s_5, s_{13}, s_{24}) \nonumber \\
    \Delta_3^{(3)} &= \lambda(s_5, s_{23}, s_{14})\,,
\end{align}
where $\lambda(a,b,c) = a^2+b^2+c^2 - 2ab-2ac-2bc$ is the Källén function.
\begin{equation} \label{eq:delta5}
    \Delta_5 = \det G(p_1, p_2, p_3, p_4)
\end{equation}
\begin{equation} \label{eq:delta5tr5}
    \Delta_5 = \tr_5^2
\end{equation}

\begin{equation} \label{eq:sijs}
    \vec{s}_5 = \{s_{12}, s_{23}, s_{34}, s_{45}, s_{15}, s_5\}\,.
\end{equation}

\subsection{Divergences and dimensional regularisation} \label{sec:divergences}
Explain integrals over loop momenta, divergences, $(d=4-2\epsilon)$-dimensional integrals, universal IR Catani structure, UV structure


\section{Divergences}
The dimensionally regulated Feynman diagrams satisfy the following properties~\cite{Wilson1973}:
\begin{itemize}
\begin{subequations}
    \item Linearity:
    \begin{equation} \label{eq:linearity}
        \int \dd^d k\, \left[ a f_1(k) + b f_2(k) \right] = a \int \dd^d k\, f_1(k) + b \int \dd^d k\, f_2(k)\,,
    \end{equation}
    where $a, b$ are arbitrary constants.
    \item Translational invariance:
    \begin{equation} \label{eq:translationalinvariance}
        \int \dd^d k\, f(k+q) = \int \dd^d k\, f(k)\,,
    \end{equation}
    where $q$ is an arbitrary vector that can depend on the external momenta $p$, as well as the loop momenta $k$ (or both).
    \item Scaling:
    \begin{equation} \label{eq:scaling}
        \int \dd^d k\, f(\lambda k) = \lambda^{-d} \int \dd^d k\, f(k)\,,
    \end{equation}
    where $\lambda$ is an arbitrary positive constant.
\end{subequations}
\end{itemize}
\end{document}