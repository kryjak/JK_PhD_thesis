\documentclass[main.tex]{subfiles}
\begin{document}
\chapter{Introduction} \label{sec:intro}
\begin{figure}[h]
    \centering
    \begin{tikzpicture}
        \randuck
    \end{tikzpicture}
    \caption{A duck with a label after caption.}
    \label{fig:after}
    This is referencing Fig.~\ref{fig:after}.
\end{figure}
\section{QED} \label{sec:QEDintro}
The central equation of quantum field theory which describes all spin\=/1/2 particles is the Dirac equation:
\begin{equation} \label{eq:Dirac}
    (\slashed{p}-m)\psi(x) = 0\,.
\end{equation}
The solutions to this equation are given by the four-component Dirac spinors $\psi(x)$, which transform in the irreducible $\left(\frac{1}{2}, 0\right) \otimes \left(0, \frac{1}{2}\right)$ representation\footnote{Not to be confused with the usual four-vectors $x^\mu$, which transform in the $\left(\frac{1}{2}, \frac{1}{2}\right)$ representation.}. There are four plane-wave solutions to the Dirac equation, which admit both positive and negative energies:
\begin{equation}
    \psi(x) = u^s(p)\ex{-\i p\cdot x} \qquad \text{and} \qquad \psi(x) = v^s(p)\ex{+\i p\cdot x}\,,
\end{equation}
where $s=\{+, -\}$ labels the spin of the fermion. The momentum space spinors $u(p)$ and $v(p)$ are interpreted as describing particles and antiparticles with \textit{positive} energies, respectively. They satisfy the momentum space Dirac equations: 
\begin{align}
    (\slashed{p} - m)u(p) &= 0\,, \\
    (\slashed{p} + m)v(p) &= 0\,,
\end{align}
as well as the following completeness relations, which is useful when performing spin sums in amplitude computations:
\begin{align} \label{eq:completeness:fermions}
    \sum_{s=\pm} u^s(p)\bar{u}^s(p) &= \slashed{p} + m\,, \\
    \sum_{s=\pm} v^s(p)\bar{v}^s(p) &= \slashed{p} - m\,.
\end{align}
Moreover, it is convenient to split the spinors into two separate components:
\begin{equation}
    \psi(x) = 
    \begin{pmatrix}
        \psi_L(x) \\
        \psi_R(x)
    \end{pmatrix}\,,
\end{equation}
where the left- and right-handed Weyl spinors transform under the $\left(\frac{1}{2}, 0 \right)$ and $\left(\frac{1}{2}, 0 \right)$ representations, respectively. Crucially, in the absence of mass, the two components fully decouple \JK{maybe a better word?}. It is also useful to define an additional $\gamma^5$ matrix (also known as the chirality operator): 
\begin{equation}
    \gamma^5 \equiv \i \gamma^0 \gamma^1 \gamma^2 \gamma^3\,,
\end{equation}
which in the Weyl (chiral) representation of the $\gamma$-matrices is just:
\begin{equation}
    \gamma^5 = 
    \begin{pmatrix}
        1 & 0 \\
        0 & -1
    \end{pmatrix}\,.
\end{equation}
Then, the Weyl spinors can be extracted from the Dirac spinors by applying appropriate projectors, according to $\psi_L(x) = P_L\psi(x)$ and $\psi_R(x) = P_R\psi(x)$:
\begin{equation}
    P_L = \frac{1-\gamma^5}{2} = 
    \begin{pmatrix}
        0 & 0 \\
        0 & 1
    \end{pmatrix}\,,
    \qquad
    P_R = \frac{1+\gamma^5}{2} = 
    \begin{pmatrix}
        1 & 0 \\
        0 & 0
    \end{pmatrix}\,.
\end{equation}
Finally, we can associate $\psi_L$ and $\psi_R$ with the \textbf{helicity} of a particle, which is the projection of its spin onto the direction of motion. The corresponding helicity values for left- and right-handed particles are $-1$ and $+1$, respectively\footnote{Strictly speaking, the left- and right-handed Weyl spinors are eigenstates of the chirality operator $\gamma^5$. However, for $m=0$, helicity and chirality eigenstates coincide}.
Weyl spinors will play a key role in the discussion of the spinor-helicity formalism in Section~\ref{sec:spinhelform}.

\section{Quantum Chromodynamics} \label{sec:QCD}
\begin{equation} \label{eq:liealgebra}
    [T^{a_1}, T^{a_2}] = \i f^{a_1a_2a_3} T^{a_3} \,.
\end{equation}
Working in the fundamental representation with normalisation $\tr(T^{a_1} T^{a_2}) = \mathrm{T}_F \, \delta^{a_1a_2}$, we multiply both sides by $T^{a_3}$ and take the trace:
\begin{align} \label{eq:f(abc)}
    \tr(T^{a_3}[T^{a_1}, T^{a_2}]) &= \i f^{a_1a_2a_4} \tr(T^{a_3} T^{a_4}) \nonumber \\
    \tr(T^{a_3} T^{a_1} T^{a_2} - T^{a_3} T^{a_2} T^{a_1}) &= if^{a_1a_2a_4} \mathrm{T}_F \delta^{a_3a_4} \nonumber \\ 
    \implies f^{a_1a_2a_3} &= -\frac{\i}{\mathrm{T}_F} \tr(T^{a_3} T^{a_1} T^{a_2} - T^{a_3} T^{a_2} T^{a_1}) \nonumber \\
    &= -\frac{\i}{\mathrm{T}_F} \tr(T^{a_1} T^{a_2} T^{a_3} - T^{a_1}T^{a_3} T^{a_2}) \,.
\end{align}

Completeness relations for the spin-1 polarisation vectors:
\begin{equation} \label{eq:completeness:bosons}
    \sum_{s=\pm} \varepsilon_\mu^s(p, q) (\varepsilon_\nu^s(p, q))^{\ast} = -g_{\mu\nu} + \frac{p_\mu q_\nu + p_\nu q_\mu}{p \cdot q}
\end{equation}
\JK{is this true only in the axial gauge? See Eq. 30 of `Calculating scattering amplitudes efficiently'. We don't use the axial gauge.}

\section{Scattering amplitudes}
\subsection{Kinematics} \label{sec:kinematics}
\begin{align} \label{eq:delta3}
    \Delta_3^{(1)} &= \lambda(s5, s12, s34) \nonumber \\
    \Delta_3^{(2)} &= \lambda(s5, s13, s24) \nonumber \\
    \Delta_3^{(3)} &= \lambda(s5, s23, s14)\,,
\end{align}
where $\lambda(a,b,c) = a^2+b^2+c^2 - 2ab-2ac-2bc$ is the Källén function.
\begin{equation} \label{eq:delta5}
    \Delta_5 = \det G(p_1, p_2, p_3, p_4)
\end{equation}
\begin{equation} \label{eq:delta5tr5}
    \Delta_5 = tr_5^2
\end{equation}
\end{document}