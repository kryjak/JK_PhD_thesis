\documentclass[main.tex]{subfiles}
\begin{document}
\chapter{Tools for calculating scattering amplitudes}
In the remainder of this thesis, we focus on various aspects of the computation of two-loop QCD scattering amplitudes for high-multiplicity processes. It is important to note that there is currently no one-size-fits-all approach that would allow us to compute all the desired amplitudes at a press of a button. In practice, we use a collection of methods that are most appropriate for the task at hand. For processes at the limit of current capabilities, these tools need to be further improved or replaced with novel ideas. To this end, much work has been done by the theory community in recent years. Unfortunately, due to the overwhelming algebraic and analytic complexity, many calculations still present challenges beyond the reach of current technology. 
\begin{figure}[t]
\begin{tikzpicture}
	\node (feynman) [greenrec] {Feynman diagrams};
	\node (colour)  [redrec,right of=feynman,xshift=4cm] {Colour decomposition};
	\node (topos)   [redrec,right of=colour,xshift=4cm] {Collect in topologies};
	\node (reduction) [bluerec,below of=topos,yshift=-2cm] {\parbox{0.3\textwidth}{\centering Integrand reduction onto \\ maximal topologies}};
	\node (IBPs)    [bluerec,left of=reduction,xshift=-4cm] {IBP reduction};
	\node (spfns)   [bluerec,left of=IBPs, xshift=-4cm] {\parbox{0.3\textwidth}{\centering Expansion of MIs onto \\ special function basis}};
	\node (sub)     [bluerec,below of=spfns,yshift=-2cm] {Pole subtraction};
	\node (finrem)  [bluerec,right of=sub,xshift=4cm] {Finite remainder};
	\node (rec)  [bluerec,right of=finrem,xshift=4cm] {Reconstruction};
	\node (qgraf) [below of=feynman, opacity=0.7] {\textcolor{green}{\Large QGRAF}};
	\node (mma) [below right of=colour, xshift=1.5cm, yshift=-0.3cm, opacity=0.7] {\textcolor{red}{\Large Mathematica/FORM}};
	\node (ff) [below of=IBPs, yshift=-0.5cm,opacity=0.7] {\textcolor{blue}{\Large finite fields}};
		
	\draw[->] (feynman.east) -- (colour.west);
	\draw[->] (colour.east) -- (topos.west);
	\draw[->] (topos.south) -- (reduction.north) node[midway,right] {$d=4-2\epsilon$};
	\draw[->] (reduction.west) -- (IBPs.east);
	\draw[->] (IBPs.west) -- (spfns.east);
	\draw[->] (spfns.south) -- (sub.north);
	\draw[->] (sub.east) -- (finrem.west) node[midway,above] {$\epsilon \rightarrow 0$};
	\draw[->] (finrem.east) -- (rec.west);
\end{tikzpicture}
\caption{A schematic overview of the workflow we adopt to compute scattering amplitudes.}
\label{fig:outline}
\end{figure}
The goal of this chapter is to provide an overview of the method we adopt in amplitude computations, as well as the problems that invariably follow. The procedure involves several highly non-trivial steps. To help the reader retain the `big picture' of the workflow, we present its schematic outline in Fig.~\ref{fig:outline}.  Each step is discussed in more detail below. 

\section{Feynman diagrams}
The starting point of our amplitude computation for a given process is the generation of all Feynman diagrams contributing to this process at the desired loop order. Feynman diagrams provide a pictorial representation of the ways in which the interaction can occur.
%time-ordered correlation functions that contribute to the $S$-matrix element
At the same time, the corresponding mathematical expressions can be easily recovered using Feynman rules, which can be derived form the Lagrangian of the theory under consideration \textcolor{red}{(Feynman rules relevant to our work are listed in Appendix~\ref{app:feynmanrules})}. As such, these diagrams are an indispensable tool of any perturbative calculation. In practice, it can be observed that usually their number grows faster than exponentially as we increase the loop order or multiplicity (see Table~\ref{tab:ndiags} for an example). To handle the combinatorial complexity, we generate the relevant Feynman diagrams using $\texttt{QGRAF}$~\cite{Nogueira:1991ex}. \textcolor{red}{This programme has the advantage of being able to control the total power of the coupling constant in a certain diagram... - should I even mention that?}
\begin{table}[b]
	\begin{center}
		\begin{tabular}{r|c|c|c|c|c|c|c|c}
			  $n$ & 1 & 2 & 3 & 4 & 5 & 6 & 7 & 8 \\
			\hline
			$n$ gluons   & -- & -- & 1 & 4 & 25 & 220 & 2485 & 34300 \\
			$q\bar{q} + n$ gluons & 1 & 3 & 16 & 123 & 1240 & 15495 & 231280 & 4016775 \\
		\end{tabular}
	\end{center}
 \caption{Number of tree-level diagrams contributing to selected processes with $n$ gluons.}
 \label{tab:ndiags}
\end{table}
\section{Colour decomposition}
Having generated the Feynman diagrams, we substitute the Feynman rules for the propagators and vertices using \texttt{Mathematica}. At this point, our QCD amplitude contains both colour and kinematic information. The idea of colour ordering is to reorganise the amplitude such that these two components separate: a purely kinematic amplitude is multiplied by the corresponding colour factor. In other words, we perform the decomposition of the full amplitude in colour space, according to a chosen colour basis. Roughly speaking:
\begin{equation}
    \ampl{n}{} = \sum_i \text{(colour)}_i \times A_{n\,i} \,, 
\end{equation}
where $A_{n\,i}$ are the colour-ordered amplitudes (also known as colour-stripped or partial amplitudes). The motivation behind this decomposition is that the colour-ordered amplitudes turn out to be significantly simpler to calculate.

The choice of the colour basis is not unique. We adopt the decomposition according to traces of the $SU(N_c)$ generators in the fundamental representation. As an example, let's look at the 4-gluon scattering at tree-level:
\begin{equation} \label{eq:3grule}
\feynmandiagram [baseline = (i.base), horizontal = i to j] {
    a1 [particle={$A^{a_1}_{\mu_1}$}] -- [gluon, momentum=$p_1$] i;
    a2 [particle={$A^{a_2}_{\mu_2}$}] -- [gluon, momentum=$p_2$] i;
    i -- [gluon] j;
    a3 [particle={$A^{a_3}_{\mu_3}$}] -- [gluon, momentum=$p_3$] j;
    a4 [particle={$A^{a_4}_{\mu_4}$}] -- [gluon, momentum=$p_4$] j;
    };
    \xrightarrow{colour}
    f^{a_1a_2b}f^{ba_2a_3} \,.
\end{equation}
The colour factor of this diagram can be expressed in terms of the generators using Eqs.~\ref{eq:liealgebra} and ~\ref{eq:fierz}:
\begin{equation}
    f^{a_1a_2b}f^{ba_2a_3} = -\frac{1}{T_F}\left(\tr[T^{a_1} T^{a_2} T^{a_3} T^{a_4}] - \tr[T^{a_1} T^{a_2} T^{a_4} T^{a_3}] - \tr[T^{a_1} T^{a_3} T^{a_4} T^{a_2}] + \tr[T^{a_1} T^{a_4} T^{a_3} T^{a_3}] \right)\,.
\end{equation}
The colour factors of the $t$- and $u$-channels can be expressed in a similar way. Combining the contributions from the three channels and using the cyclicity of the trace, we can organise the 4-gluon amplitude at tree-level as follows:
\begin{equation}
    \ampl{4}{(0)} =  g^2 \left( \tr[T^{a_1} T^{a_2} T^{a_3} T^{a_4}]\,A_{4}^{(0)}[1234] + \text{permutations of } (234) \right) \,.
\end{equation}
In the general case, this formula reads:
\begin{equation} \label{eq:colour-decomposition}
    \ampl{n}{(0)} = g^{n-2} \sum_{\sigma \in S_{n-1}} \tr\left[T^{a_1} T^{\sigma(a_2} \ldots T^{a_n)}\right] A_n^{(0)} \left[1\,\sigma(2\ldots n)\right] \, ,
\end{equation}
where the sum is over the set of all \textit{non-cyclic} permutations of $n-1$ particles. Similar colour decompositions can be derived for amplitudes involving quarks, as well as beyond tree-level~\cite{Dixon:1996wi}. At loop-level, the colour basis contains products of traces, in addition to single trace structures. 

The colour-ordered amplitudes are calculated by considering the kinematic part of all Feynman diagrams contributing to a given colour factor. They are gauge invariant and satisfy a number of important identities:
\begin{align}
    A_n[123\ldots n] &= A_n[23\ldots n\,1]\,, && \text{cyclicity} \\
    A_n[123\ldots n] &= (-1)^n A_n[n \ldots 231] \,, && \text{reflection} \\
    A_n[123 \ldots n] &+ A_n[213 \ldots n] +  A_n[231 \ldots n] +  \ldots + && A_n[23 \ldots 1\,n]  = 0 \,, \nonumber \\ 
    & && U(1) \text{ decoupling} \\
    A_n[1, {\alpha}, n, {\beta}] &= (-1)^{|\beta|} \sum_{\mathclap{\sigma \in OP(\{\alpha\} \cup \{\beta^T\})}} A_n[1, \sigma, n] \,, && \text{Kleiss-Kuiff relations}
\end{align}
\textcolor{red}{fix alignment!!!} where the sum is over permutations in the joint set $\{\alpha\} \cup \{\beta\}$ such that the order within the individual sets is preserved, and $\{\beta^T\}$ is the reversal of the set $\{\beta\}$~\cite{Mangano:1990by, Kleiss:1989616, Bern:2008qj}. Crucially, these properties allow us to re the number of independent helicity amplitudes, to which we turn our attention next.

\section{Helicity amplitudes}
\textcolor{red}{According to this approach, one calculates matrix elements with external states having a given assigned helicity. Since different helicity configurations do not interfere, to obtain the full cross
section it is sufficient to sum incoherently the squares of all of the possible helicity amplitudes
which can contribute to the process. The advantage over more standard techniques is that by
choosing a definite helicity configuration one can exploit gauge invariance and select an explicit
representation for the polarization vectors which will simplify the calculation.}
\end{document}