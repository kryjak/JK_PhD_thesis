\section{Definition of the Feynman integral families}
\label{app:int_def}

For each two-loop integral family $\tau$ corresponding to one of the maximal topologies shown in \cref{fig:int-fams},
the Feynman integrals have the form
\begin{equation}
	j^{\tau}(a_1, \ldots, a_9) = \mathrm{e}^{2 \eps \gamma_{E}} \int \frac{\mathrm{d}^{4-2\eps} k_1}{\mathrm{i} \pi^{2-\eps}} \frac{\mathrm{d}^{4-2\eps} k_2}{\mathrm{i} \pi^{2-\eps}} \frac{1}{D_{\tau,1}^{a_1} \ldots D_{\tau,9}^{a_9}} \,.
\end{equation}
The sets $\{D_{\tau,1}, \ldots, D_{\tau,9}\}$ contain seven (inverse) propagators and two \acp{ISP} ($a_8, a_9 \le 0$). 
For the maximal topologies under consideration, they are given by:\footnote{We use a naming convention analogous to that of \incite{Abreu:2020jxa}.}
\begin{itemize}
	\item penta-triangle, \textbf{mzz} configuration:
		\begin{align} \begin{aligned}
			\big\{k_1^2,(k_1+p_1+p_2+p_3)^2,(k_1+p_2+p_3)^2,(k_1+p_3)^2,k_2^2,(k_2-p_3)^2,\\(k_1+k_2)^2,(k_2-p_1-p_2-p_3)^2,(k_2-p_2-p_3)^2\big\} \,,
		\end{aligned} \end{align}
	\item penta-triangle, \textbf{zmz} configuration:
		\begin{align} \begin{aligned}
		\big\{k_1^2,(k_1-p_1)^2,(k_1+p_2+p_3)^2,(k_1+p_3)^2,k_2^2,(k_2-p_3)^2,(k_1+k_2)^2,\\(k_2+p_1)^2,(k_2-p_2-p_3)^2 \big\} \,,
		\end{aligned} \end{align}
	\item penta-triangle, \textbf{zzz} configuration:
		\begin{align} \begin{aligned}
			\big\{k_1^2,(k_1-p_1)^2,(k_1-p_1-p_2)^2,(k_1-p_1-p_2-p_3)^2,k_2^2,(k_2+p_1+p_2+p_3)^2, \\ (k_1+k_2)^2,(k_2+p_1)^2,(k_2+p_1+p_2)^2\big\} \,,
		\end{aligned} \end{align}
	\item planar double-box:
		\begin{align} \begin{aligned}
			\big\{k_1^2,(k_1-p_1)^2,(k_1-p_1-p_2)^2,k_2^2,(k_2+p_1+p_2+p_3)^2,(k_2+p_1+p_2)^2, \\ (k_1+k_2)^2,(k_1-p_1-p_2-p_3)^2,(k_2+p_1)^2\big\} \,,
		\end{aligned} \end{align}
	\item crossed double-box, \textbf{mz} configuration:
		\begin{align} \begin{aligned}
			\big\{k_1^2,(k_1+p_1+p_2+p_3)^2,(k_1+p_2+p_3)^2,k_2^2,(k_2-p_2)^2,(k_1+k_2)^2,\\ (k_1+k_2+p_3)^2,(k_1+p_3)^2,(k_2-p_1-p_2-p_3)^2\big\} \,,
		\end{aligned} \end{align}
	\item crossed double-box, \textbf{zz} configuration:
		\begin{align} \begin{aligned}
			\big\{k_1^2,(k_1-p_1)^2,(k_1-p_1-p_2)^2,k_2^2,(k_2-p_3)^2,(k_1+k_2)^2, \\ (k_1+k_2-p_1-p_2-p_3)^2,(k_1-p_1-p_2-p_3)^2,(k_2+p_1)^2\big\} \,.
		\end{aligned} \end{align}
\end{itemize}
%
We also use the one-loop (one-mass) box family, made of the following integrals:
\begin{equation}
	j^{\rm box}(a_1, a_2, a_3, a_4) = \mathrm{e}^{\eps \gamma_{E}} \int \frac{\mathrm{d}^{4-2\eps} k}{\mathrm{i} \pi^{2-\eps}} \frac{1}{D_{\mathrm{box}, 1}^{a_1} D_{\mathrm{box},2}^{a_2} D_{\mathrm{box},3}^{a_3} D_{\mathrm{box},4}^{a_4}} \,,
\end{equation}
with the four inverse propagators $D_{\mathrm{box},i}$
	\begin{equation}
		\big\{k_1^2, (k_1-p_1)^2, (k_1-p_1-p_2)^2, (k_1-p_1-p_2-p_3)^2 \big\} \,.
	\end{equation}
Feynman's prescription for the imaginary parts of all propagators is implicit.

These family definitions (strictly with the ordering of inverse propagators and \acp{ISP} shown above) correspond to the integrals \texttt{j[family,$a_1,$\ldots]} that build the canonical \ac{MI} bases provided in the \texttt{pure\_mi\_bases/} directory of our ancillary files~\cite{zenodo}. In this notation, each \texttt{j[...]} represents a Feynman integral within a given integral family, while the numbers $a_i$ refer to the powers of its propagators and \acp{ISP}.

\section{Optimised \acs{IBP} reduction procedure for amplitudes with many permuted integral families}
\label{app:altIBPs}
An amplitude will in general have contributions from permutations of
the \textit{ordered} integral families shown in figure \ref{fig:int-fams}. To reduce the tensor integrals in the amplitude, \ac{IBP} identities must
be generated for all the permutations of these ordered families. This can
lead to a very large \ac{IBP} system. 
The performance of the reduction setup is extremely sensitive to the number of \ac{IBP}
identities required so, to minimise the memory consumption, we choose to
generate \ac{IBP} identities only for the ordered families. Next, we obtain the reduction for any permutation of these families by permuting the `ordered' reduction numerically over finite fields.
The result is then given in terms of \acp{MI} of each family permutation, but it is missing the symmetry relations that can be found between subsectors of different families. To express the final result in terms of a minimal set of \acp{MI}, we find such relations
from a separate computation. One may account for integral symmetries using automated tools such as \texttt{LiteRed}~\cite{Lee:2012cn}. Since we use a pure basis of \acp{MI}, the symmetry relations
amongst them will have rational numbers as coefficients. This is because the presence of any kinematic invariant would spoil the purity of the canonical \acp{DE} (see \cref{sec:spec-fns}), and would mean that such a symmetry relation in fact involves non-pure integrals. Therefore, the computation of the missing symmetry relations can be performed with all kinematic invariants set to numeric values, which significantly lowers the complexity of this task. Finally, we note that even if symmetries amongst the
\acp{MI} were missed, a representation of the integrals in terms of a basis of
special functions ---~as we construct in \cref{sec:spec-fns}~--- would automatically incorporate the extra simplifications and
so the same final result would be obtained. Nonetheless, in practice we do find it useful to include these symmetry relations, as they reduce the number of independent coefficients that have to be processed further.

The procedure can be summarised as follows:
\begin{enumerate}
	\item Generate (analytic) IBPs for the six ordered families.
	\item Compute the mappings between permutations of the \acp{MI} of the system above.
	\item Take the tensor integrals in the amplitudes for each permutation of these families and solve the linear system over finite fields.
	\item Apply the symmetry mappings between the \acp{MI} of each family permutation to find the minimal set for the full system.
\end{enumerate}

Since there are a few additional bits of terminology, we can consider a concrete
example to clarify everything. At one-loop, a four-point process with a
single off-shell leg can be described by a single independent integral family
which is simply the box topology (see \cref{app:int_def} for its explicit definition). Following the Laporta reduction algorithm leads to a basis
of four \acp{MI},
\begin{equation}
	{\rm MI}^{\rm box} = \{j^{\rm box}(1,1,1,1),\, j^{\rm box}(1,0,1,0),\, j^{\rm box}(0,1,0,1),\, j^{\rm box}(1,0,0,1)\} \,,
\end{equation}
which are the scalar box and scalar bubble integrals in channels $s_{12},
s_{23}$ and $s_4$ respectively. An amplitude will, in general, be written in
terms of three permutations of this family. Let us denote
these permutations as $j^{\rm box, 1234}$, $j^{\rm box, 2314}$, and $j^{\rm box, 3124}$,
where $j^{\rm box, 1234} = j^{\rm box}$ as above and the additional superscript indices refer to the order of the external legs. Following our procedure we
would load one set of IBP relations generated for $j^{\rm box}$. These
identities can then be permuted numerically, for example as \texttt{FiniteFlow} graphs, to reduce
tensor integrals in each of the three permuted families. The result is now in terms
of twelve \acp{MI}: three boxes and nine bubbles. While the amplitude is already
in a minimal basis of box integrals, there is clearly an over-complete set of
bubbles. The independent bubbles are in the channels $s_{12}$, $s_{23}$,
$s_{13}$, and $s_4$, so the five additional symmetry mappings are
\begin{align}
	\begin{aligned}
		j^{\rm box, 2314}(1,0,1,0) &= j^{\rm box, 1234}(0,1,0,1) \,, &
		j^{\rm box, 3124}(1,0,1,0) &= j^{\rm box, 2314}(0,1,0,1) \,, \\
		j^{\rm box, 3124}(0,1,0,1) &= j^{\rm box, 1234}(1,0,1,0) \,, &
		j^{\rm box, 2314}(1,0,0,1) &= j^{\rm box, 1234}(1,0,0,1) \,, \\
		j^{\rm box, 3124}(1,0,0,1) &= j^{\rm box, 1234}(1,0,0,1) \,.
	\end{aligned}
\end{align}
After applying these identities we arrive at the final result with seven
\acp{MI} which cover all permutations of the integral families. 
This approach would not lead to any significant performance
enhancements in this simple example of course,
but it can be particularly important when considering
high-multiplicity examples where the number of permutations is
high.

\section{Rational parametrisation of the kinematics}
\label{app:mtvs}

Since we are applying finite-field techniques to helicity amplitudes, we employ a rational parametrisation of the external kinematics using Hodges's momentum twistor formalism~\cite{Hodges:2009hk}.
While this is not essential to combat the algebraic complexity for the kinematics considered here, it does provide a convenient parametrisation of the spinor products.

The single-off-shell four-particle phase space $p$ is obtained from a massless five-particle parametrisation $q$ (defined in Appendix A of~\incite{Badger:2021imn} with $\{x_2\leftrightarrow x_4,x_3\leftrightarrow x_5\}$) under
\begin{align}
    p_i &= q_i \qquad \forall i=1,2,3, & p_4=q_4+q_5 \, .
\end{align}
The momentum twistor variables $x_i$ for $p$ are then related to the scalar invariants $\vec{s}$ through
\begin{align} \label{eq:mtvs}
    s_{12} &= x_1 \,, &
    s_{23} &= x_1 x_2 \,, &
    s_4 &= x_1 x_3 \,.
\end{align}

Momentum twistors allow us to express any spinor expression as a rational function in the variables $x_i$.
In this representation the helicity scaling is however obscured, as we have fixed the spinor phases in order to achieve a parameterisation in terms of the minimal number of variables (see e.g.~\incite{Badger:2016uuq}).
Therefore, we need to manually restore the phase information at the end of the computation.
This can be achieved by multiplying the momentum twistor expression by an arbitrary factor $\Phi$ with the same helicity scaling as the helicity amplitude under consideration, divided by that factor written in terms of momentum twistor variables.
For example, for the helicity configurations of \cref{eq:helconfs}, we can use the phase factors
\begin{align}
  \Phi(-++) &= \frac{\langle 1 2 \rangle}{\langle 2 3 \rangle^2} \,,&
  \Phi(-+-) &= \frac{[ 1 2 ]}{[ 1 3 ]^2} \,,
\end{align}
which in our momentum twistor parameterisation are given by
\begin{align}
  \Phi(-++) &= x_1^2 \,,&
  \Phi(-+-) &= - \frac{1}{x_1 (1 + x_2 - x_3)^2}   \,.
\end{align}
We refer to Appendix~C of \incite{Badger:2023mgf} a thorough discussion of how to restore the phase information in a momentum twistor parameterisation.


\section{Renormalisation and \acl{IR} structure}
\label{app:poles}

We renormalise the coupling constant by trading the bare coupling $\alpha_{\mathrm{bare}}$ for the renormalised one $\alpha_{\mathrm{R}}$ through
\begin{align}
    \label{eq:alpha-bare}
    \alpha_{\mathrm{bare}} = \alpha_{\mathrm{R}}(\mu_R) \, Z_{\alpha}\bigl(\alpha_{\mathrm{R}}(\mu_R) \bigr)  \, \mu_R^{2 \eps} \, S_{\eps} \,,
\end{align}
with $S_\eps=(4\pi)^{-\eps} \mathrm{e}^{\eps\gamma_E}$.
The renormalisation factor $Z_{\alpha}$ in the $\overline{\text{MS}}$ scheme is \cite{Barnreuther:2013qvf,Bonciani:2021okt}
\begin{align}
    \label{eq:Za}
    Z_{\alpha}(\alpha) = 1 - \frac{\alpha}{4 \pi} \frac{\beta_0}{\eps} -
    \left(\frac{\alpha}{4 \pi}\right)^2 \left( -\frac{\beta_0^2}{\eps^2} + \frac{1}{2}\frac{\beta_1}{\eps} \right) + \mathcal{O}\bigl(\alpha^3\bigr) \,.
\end{align}
The $\beta$-function is defined from the renormalised coupling as
\begin{align} \label{eq:beta_func_def}
\frac{\dd \alpha_{\rm R}(\mu_R)}{\dd \ln \mu_R} = \left[ -2 \, \eps + \beta\bigl(\alpha_{\rm R}(\mu_R)\bigr) \right]  \alpha_{\rm R}(\mu_R) \,,
\end{align}
and expanded as
\begin{align} \label{eq:beta_func}
\beta( \alpha ) = -2 \, \frac{\alpha}{4 \pi}  \sum_{k\ge 0} \beta_k \left(\frac{\alpha}{4\pi} \right)^k \,,
\end{align}
with 
\begin{align} \label{eq:beta_coeffs}
\beta_0 = -\frac{4}{3} \nl \,, \qquad \qquad \beta_1 = - 4 \, \nl \,.
\end{align}
The photon wavefunction renormalisation factor is $Z_A=Z_\alpha$, which we include due to the external off-shell photon.
The complete renormalisation procedure then is
\begin{align}
    \mathcal{A}^\mu_\text{renorm}(\alpha_R) = Z_A^{\frac{1}{2}}(\alpha_R) \, \mathcal{A}^\mu_\text{bare}(\alpha_\text{bare}) \, ,
\end{align}
where $\alpha_\text{bare}$ is expressed in terms of $\alpha_R$ through \cref{eq:alpha-bare}.

The \ac{IR} poles of the renormalised amplitude factorise as~\cite{Catani:1998bh,Gardi:2009qi,Gardi:2009zv,Becher:2009cu,Becher:2009qa}
\begin{align}
 \mathcal{A}^\mu_\text{renorm}(\alpha_R) = Z(\alpha_R) \,  \mathcal{F}^\mu(\alpha_R) \,,
\end{align}
so that $ Z(\alpha_R)$ captures all \ac{IR} poles and $\mathcal{F}^\mu$ is a finite remainder.
We obtain the explicit two-loop expression of the \ac{IR} factor $Z(\alpha_R)$ by
choosing \ac{QED} parameters ($C_A=0$, $C_F=1$, and $T_F=1$) in the non-abelian gauge-theory expressions of \incite{Becher:2009qa}. We expand it as
\begin{align}
    Z(\alpha) &= \sum_{k\ge0} Z^{(L)} \left(\frac{\alpha}{4\pi}\right)^L \,.
\end{align}
The coefficients $Z^{(L)}$ are expressed in terms of the anomalous dimension
\begin{align}
    \Gamma &= \gamma^\text{cusp}\ln\left(\frac{-s_{12}}{\mu^2}\right)+2\gamma^l+\gamma^A \,,
\end{align}
and its derivative
\begin{align}
    \Gamma^\prime &\coloneqq \frac{\partial\Gamma}{\partial\ln\mu} = -2\gamma^\text{cusp} \,.
\end{align}
Here, $\gamma^\text{cusp}$ is the cusp anomalous dimension, while $\gamma^l$ and $\gamma^A$ are the lepton's and the photon's collinear anomalous dimensions, respectively.
We expand all anomalous dimensions $y\in\{\Gamma,\gamma^i\}$ as
\begin{align}
    y &= \frac{\alpha}{4\pi} \sum_{k\ge0} y_k \left(\frac{\alpha}{4\pi}\right)^k \,,
\end{align}
with coefficients
\begin{subequations}
\begin{align}
    \gamma^l_0 &= -3 \,, &
    \gamma^l_1 &= -\frac{3}{2}+2\pi^2-24 \, \zeta_3+\nl\left(\frac{130}{27}+\frac{2}{3}\pi^2\right) \,, \\
    \gamma^A_0 &= -\beta_0 \,, &
    \gamma^A_1 &= -\beta_1 \,, \\
    \gamma^\text{cusp}_0 &= 4 \,, &
    \gamma^\text{cusp}_1 &= -\frac{80}{9} \nl \,.
\end{align}
\end{subequations}
Finally, the coefficients of the \ac{IR} factor $Z$ up to two loop are given by
\begin{align}
    \label{eq:ir-pole-coeffs}
    Z^{(0)} &= 1 \,, &
    Z^{(1)} &= \frac{\Gamma_0^\prime}{4\eps^2}+\frac{\Gamma_0}{2\eps} \,, &
    Z^{(2)} &= \frac{{Z^{(1)}}^2}{2} -\frac{3\beta_0\Gamma_0^\prime}{16\eps^3}+\frac{\Gamma_1^\prime-4\beta_0\Gamma_0}{16\eps^2}+\frac{\Gamma_1}{4\eps} \,.
\end{align}

Putting together the subtraction of \ac{UV} and \ac{IR} poles, and expanding the resulting finite remainder $\mathcal{F}^{\mu}(\alpha_R)$ in $\alpha_R$ leads to the definitions in \cref{eq:finite-remainders}.

\section{Analytic continuation}
\label{app:an_cont}

We analytically continue the \acp{MPL} by adding a small positive (or negative) imaginary part to the \ac{MPL} indices $l_i$ in \cref{eq:indices} whenever they fall between $0$ and $1$. The imaginary part of each index prescribes how to deform the integration contour around the pole associated with it. We do similarly for the logarithms in \cref{eq:logs}.
To this end, following \incite{Gehrmann:2002zr}, we change variables from $(s_{12},s_{23},s_4)$ to $(s_{12},s_{23},s_{13})$, with $s_4 = s_{12} + s_{23} + s_{13}$. We then add a small positive imaginary part to the latter variables, as
\begin{align} \label{eq:add_im}
s_{12} \longrightarrow s_{12} + \mathrm{i} \, c_{1} \, \delta \, , \qquad
s_{23} \longrightarrow s_{23} + \mathrm{i} \, c_{2} \, \delta  \, , \qquad
s_{13} \longrightarrow s_{13} + \mathrm{i} \, c_{3} \, \delta  \, , 
\end{align}
where $c_{1}$, $c_{2}$ and $c_{3}$ are arbitrary positive constants, and $\delta$ is a positive infinitesimal. 
Finally, we check whether this substitution gives a positive or negative imaginary part to each \ac{MPL} index $l_i$.
This depends on the domain of the kinematic variables.
We focus on three kinematic regions which are of phenomenological interest. The analytic continuation for any other region may be obtained similarly.

\smallskip

\paragraph{Electron-line corrections to $e^- \mu^- \to e^- \mu^- \gamma$.}
To define the domain of the kinematic variables relevant for this application, 
we embed the four-particle off-shell process of \cref{eq:scatter} in the five-particle process $e^- \mu^- \to e^- \mu^- \gamma$. We then determine the kinematic constraints for the five-particle process (see e.g. Appendix~A of \incite{Chicherin:2021dyp}), and from them derive the constraints on the four-point off-shell kinematics. The result is
\begin{align}
\label{eq:region_emu-emugamma}
\mathcal{P}_{e\mu\to{e}\mu\gamma} \coloneqq \{\vec{s} \colon s_{12} < 0 \, \land \, s_{23} < 0 \, \land \, 0 < s_{13} < -s_{12} - s_{23} \} \,.
\end{align}
The \ac{MPL} index $l_4 = - s_{12}/s_{23}$ is always negative in $\mathcal{P}_{{e}\mu\to{e}\mu\gamma}$, hence no analytic continuation is required. The other three indices may instead fall between $0$ and $1$. Let us study $l_1$. Changing variables from $s_4$ to $s_{13}$ and adding imaginary parts as in \cref{eq:add_im} gives
\begin{align}
l_1 = \frac{s_{12} + s_{13} + s_{23}}{s_{12}} +  \frac{\mathrm{i} \delta}{s_{12}^2} \left[ (c_2 + c_3) s_{12} - c_1 (s_{13} + s_{23}) \right] + \mathcal{O}\left(\delta^2\right) \,.
\end{align}
The imaginary part of $l_1$ may be either negative or positive in $\mathcal{P}_{{e}\mu\to{e}\mu\gamma} $. However, it is strictly negative in the subregion of $\mathcal{P}_{{e}\mu\to{e}\mu\gamma} $ where $0<l_1<1$. We therefore assign a negative imaginary part to $l_1$ whenever $0<l_1<1$ in $\mathcal{P}_{{e}\mu\to{e}\mu\gamma} $. The analysis of the other indices follows similarly, and is summarised in \cref{tab:an_cont}.
The arguments of the three logarithms in \cref{eq:logs} are positive in $\mathcal{P}_{e\mu\to{e}\mu\gamma}$.

\begin{table}
    \centering
    \begin{tabular}{cccc}
        \hline
        Index & $\mathcal{P}_{{e}\mu\to{e}\mu\gamma}$ & $\mathcal{P}_{{e}\bar{{e}}\to\gamma \gamma^*}$ & $\mathcal{P}_{\gamma^*\to e \bar{e} \gamma}$  \\
        \hline
        $l_1$  & $-$ & $+$ & $0$ \\
        $l_2$  & $-$ & $0$ & $0$ \\
        $l_3$  & $-$ & $0$ & $0$ \\
        $l_4$  & $0$ & $0$ & $0$ \\
        \hline
    \end{tabular}
    \caption{Imaginary parts of the \ac{MPL} indices defined by \cref{eq:indices} in the two kinematic regions discussed in \cref{app:an_cont}. The symbol $+$ ($-$) denotes a positive (negative) imaginary part, while $0$ means no analytic continuation is needed.}
    \label{tab:an_cont}
\end{table}

\smallskip

\paragraph{Corrections to $e^- e^+ \to \gamma \gamma^*$.} The relevant domain of the kinematic variables in this case can be derived directly for the four-point kinematics, and is typically named the $s_{12}$ channel. It is given by 
\begin{align}
\label{eq:region_eebar}
\mathcal{P}_{e\bar{e}\to\gamma\gamma^*} \coloneqq \{\vec{s} \colon s_{23} < 0 \, \land \, s_{13} < 0 \, \land \, s_{12} > - s_{23} - s_{13} \} \,.
\end{align}
The \ac{MPL} indices $l_2$, $l_3$ and $l_4$ can never fall between $0$ and $1$ in $\mathcal{P}_{e\bar{e}\to\gamma\gamma^*}$, and hence require no analytic continuation. We instead need to add a positive imaginary part to $l_1$. 
In this region also the logarithms in \cref{eq:logs} need to be analytically continued. The argument of $\log(s_{12}/s_4)$ is positive in $\mathcal{P}_{e\bar{e}\to\gamma\gamma^*}$. By adding imaginary parts to the arguments of the other logarithms and studying them where the arguments are negative in $\mathcal{P}_{e\bar{e}\to\gamma\gamma^*}$, we determine that the analytic continuation is achieved through the following replacements:
\begin{align}
\log(s_{23}/s_4) \longrightarrow \log(-s_{23}/s_{4})+ \mathrm{i} \pi \,, \qquad  \log(-s_4) \longrightarrow \log(s_4)- \mathrm{i} \pi \,.
\end{align}


\paragraph{Corrections to the decay $\gamma^* \to e^- e^+ \gamma$.} The relevant domain of the kinematic variables~is
\begin{align}
\label{eq:region_decay}
\mathcal{P}_{\gamma^*\to e \bar{e} \gamma} \coloneqq  \{ \vec{s} \colon s_{12} > 0 \, \land \, s_{23} > 0 \, \land \,  s_{13} > 0 \} \,.
\end{align}
All \ac{MPL} indices $l_i$ in \cref{eq:indices} are either $l_i < 0$ or $l_i > 1$, hence no analytic continuation is required. The same holds for the first two logarithms in \cref{eq:logs}, whose arguments are positive. The only function which needs to be analytically continued is $\log(-s_4)$. We achieve this by replacing
\begin{align}
\log(-s_4) \longrightarrow \log(s_{4}) - \mathrm{i} \pi \,.
\end{align}

\smallskip

The information about the imaginary parts of the \ac{MPL} indices can be fed into the publicly available libraries for evaluating these functions numerically, such as \texttt{FastGPL}~\cite{Wang:2021imw}, \texttt{GiNaC}~\cite{Bauer:2000cp,Vollinga:2004sn}, and \texttt{handyG}~\cite{Naterop:2019xaf}. This typically leads to longer evaluation times with respect to \acp{MPL} which do not need analytic continuation. We find that this is not an issue for the planned applications of our results (see \cref{sec:performance}). Nonetheless, we note that a more performant evaluation may be achieved by tailoring the representation to the kinematic region of interest in such a way that no \acp{MPL} require analytic continuation. We refer to \incite{Gehrmann:2002zr,Gehrmann:2023etk} for a detailed discussion. 