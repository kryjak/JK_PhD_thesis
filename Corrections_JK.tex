\documentclass[main.tex]{subfiles}

\begin{document}
\begin{enumerate}
    \item \textbf{Chapter 3. Put the computation in perspective from the experimental point of view} \\
    Added more experimental context to the first paragraph of Section 3.1 (`From the perspective of phenomenology...'). Included the size of the $b\bar{b}H$ production cross section at 13 TeV. Emphasised large background to this process and cited recent work on efforts to isolate the signal using machine learning techniques. Quoted current accuracy bounds for the value of the $b$-quark Yukawa coupling in a New Physics scenario. 
    
    \item \textbf{Pg. 68 has the sentence `The presence of $b$-tagging further suppresses the bbH
production rate. Though the meaning can be reconstructed, the presence or not of
tagging does not change a rate, it changes detection} \\
Indeed, the meaning of this phrase was not clear. Changed `suppresses the production rate' $\rightarrow$ `lowers the detected rate'.

    \item \textbf{Summary of Chapter 3, sentence `To the best of our knowledge this is the first...' can be broken down and clarified to further highlight the novelty of the result}
\\
Expanded the first paragraph of Section 3.7 (`In this chapter, we have presented...') to manifestly state the new elements of our work --- the first full set of helicity amplitudes for this kinematic configuration, construction of a momentum twistor parametrisation for this kinematics, as well as an improvement on the special function basis presented in [hep-ph/2102.02516].

    \item \textbf{Chapter 4, again put in context, e.g. there seems to be one loop EW corrections that might be more relevant than those computed, are these known?} \\
Added a paragraph at the end of Section 4.6 (`Our work opens the path...') to discuss the size of the NLO EW corrections for the related processes: $W^\pm \gamma$ and $W^\pm j$ production. NLO EW corrections to $W^\pm \gamma j$ not known yet.

\item \textbf{Given the title contains “high multiplicity”, maybe comment on the Summary what
would take to tackle higher-legs amplitude and how the “amount of work” would
scale} \\
Significantly extended Chapter 6 to highlight the challenges associated with computing amplitudes of even higher multiplicity. Gave examples of how the number of Feynman diagrams scales with additional external legs. Also summarised the current state of IBP relations for various kinematic configurations and pointed out relevant results for the non-planar topologies appearing at two-loop, five-point, one-mass scattering kinematics. These results became available after the completion of the work upon which the thesis is based. Discussed problems with including non-planar topologies in terms of the size of the IBP system and mentioned recent work on simplifying these systems through syzygy relations. Finally, discussed the limits of current technology for computing analytic expressions for scattering amplitudes and suggested a possibility of instead using an algorithm that would provide robust numerical calculation of the amplitudes at chosen phase space points.

\end{enumerate}
\end{document}