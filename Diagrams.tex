\documentclass[main.tex]{subfiles}
\begin{document}
\chapter{Drawings} 

%\begin{figure}
%    \centering
%    \begin{subfigure}[b]{0.3\textwidth}
%        \hspace*{1em}\raisebox{1.1cm}{ % needed to align the diagrams
%        \begin{tikzpicture}
%        	\begin{feynman}[small]
%        		\vertex (v1);
%          
%        		\vertex[above left = 0.8cm of v1, yshift=-0.2cm] (i1);
%        		\vertex[below left =  0.8cm of v1, yshift=+0.2cm] (i2);
%        		\vertex[right = of v1] (v2);			
%        		\vertex[right = of v2] (v3);			
%        		
%        		\vertex[above right = 0.8cm of v3, yshift=-0.2cm] (f1);
%        		\vertex[right = 0.70cm of v3] (f2);
%        		\vertex[below right = 0.8cm of v3, yshift=+0.2cm] (f3);
%
%        		\diagram*{
%        			(i1) -- (v1) -- [out=60, in=120] (v2) -- [out=60, in=120] (v3) -- (f1),
%        			(i2) -- (v1) -- [out=-60, in=-120] (v2) -- [out=-60, in=-120] (v3) -- (f3),
%                    (v3) -- (f2),
%                
%        		};
%        	\end{feynman}
%        \end{tikzpicture}}
%    \caption{} \label{fig:topologya}
%    \end{subfigure}
%    \begin{subfigure}[b]{0.3\textwidth}
%        \hspace*{1em}
%        \begin{tikzpicture}
%    	\begin{feynman}[small]
%    		\vertex (v1);
%      
%    		\vertex[above left = 0.8cm of v1, yshift=-0.2cm] (i1);
%    		\vertex[below left =  0.8cm of v1, yshift=+0.2cm] (i2);
%    		\vertex[above right = of v1] (v2);			
%    		\vertex[below right = of v1] (v3);			
%    		\vertex[right = of v2] (v4);			
%    		\vertex[right = of v3] (v5);			
%    		
%    		\vertex[above right = 0.8cm of v4] (f1);
%    		\vertex[below right = 0.5cm of v5, yshift=-0.4cm] (f2);
%    		\vertex[below right = 0.5cm of v5, xshift=+0.4cm] (f3);
%    		
%    		\diagram*{
%    			(i1) -- (v1) -- (v2) -- (v4) -- (f1),
%    			(i2) -- (v1) -- (v3) -- (v5) -- (f2),
%                (v2) -- (v3),
%                (v4) -- (v5),
%                (v5) -- (f3),
%    		};
%    	\end{feynman}
%    \end{tikzpicture}
%    \caption{} \label{fig:topologyb}
%    \end{subfigure}
%    \begin{subfigure}[b]{0.3\textwidth}
%        \hspace*{1em}
%        \begin{tikzpicture}
%    	\begin{feynman}[small]
%    		\vertex (i1);
%    		\vertex[below right= 0.8cm of i1] (v1);
%    		\vertex[right = of v1] (v2);
%    		\vertex[yshift=0.3cm, right = 0.8cm of v2] (v3);
%    		\vertex[above right = 0.8cm of v3] (f1);			
%    		
%    		\vertex[below = of v1] (v7) ;
%    		\vertex[below left = 0.8cm of v7] (i2) ;
%    		\vertex[right = of v7] (v6) ;
%    		\vertex[yshift=-0.3cm, right = 0.8cm of v6] (v5) ;
%    		\vertex[below right = 0.8cm of v5] (f3) ;
%    		
%    		\vertex[xshift=0.6cm, yshift=0.2cm, below = of v3] (v4);
%    		\vertex[right = 0.8cm of v4] (f2) ;		
%    		
%    		\diagram*{
%    			(i1) -- (v1) -- (v2) -- (v3),
%    			(i2) -- (v7) -- (v6) -- (v5),
%    			(v3) -- (f1),
%    			(v3) -- (v4) -- (v5),
%    			(v4) -- (f2),
%    			(v5) -- (f3),
%    			(v1) -- (v7),
%    			(v2) -- (v6),
%    		};
%    	\end{feynman}
%    \end{tikzpicture}
%    \caption{} \label{fig:topologyc}
%    \end{subfigure}
%\caption{Examples of diagram topologies associated with five-particle Feynman diagrams. It is easy to see that topologies (a) and (b) can be obtained from topology (c) by pinching some of its propagators.}
%\label{fig:topologies}
%\end{figure}

%\begin{figure}
%    \centering
%    \begin{tikzpicture}
%
%        %axes
%        \draw[thick,->] (-3.5,0) -- (3.5,0) node[anchor=north west] {$\nu_1$};
%        \draw[thick,->] (0,-3.5) -- (0,3.5) node[anchor=south east] {$\nu_2$};
%
%        %dots and axes labels
%        \draw[fill] (0,0) circle (1pt) node[anchor=north east] {$0$};
%        \foreach \i in {-3,-2,-1,1,2,3}
%            {
%            \draw[fill] (0,\i ) circle (1pt) node[anchor=east] {$\i$};
%            \draw[fill] (\i,0) circle (1pt) node[anchor=north] {$\i$};
%            \foreach \j in {-3,-2,-1,1,2,3}
%                \draw[fill] (\j, \i) circle (1pt);
%            }
%
%        %lines separating sectors
%        \draw[thick, blue] (0.5, -3.5) -- (0.5, 3.5);
%        \draw[thick, blue] (-3.5, 0.5) -- (3.5, 0.5);
%
%        \node at (2, 2.5) {$\bm{\theta_I} = \{1, 1\}$};
%        \node at (-2, 2.5) {$\bm{\theta_{II}} = \{0, 1\}$};
%        \node at (2, -2.5) {$\bm{\theta_{III}} = \{1, 0\}$};
%        \node at (-2, -2.5) {$\bm{\theta_{IV}} = \{0, 0\}$};
%
%
%        \clip (-3.3,-3.3) rectangle (0.3, 0.3);
%        \foreach \i in {-12,...,0}
%            \draw[dashed, red, samples=100] plot function{-x+(0.3+0.5*\i)} node[right] {};
%    \end{tikzpicture}
%    \caption{Caption}
%    \label{fig:sectors}
%\end{figure}


%\begin{figure}
%    \centering
%    \begin{subfigure}[b]{0.4\textwidth}
%    \begin{tikzpicture}
%        %axes
%        \draw[thick,->] (-3.5,0) -- (3.5,0) node[anchor=north west] {$\nu_1$};
%        \draw[thick,->] (0,-3.5) -- (0,3.5) node[anchor=south east] {$\nu_2$};
%        %dots and axes labels
%        \draw[fill] (0,0) circle (1pt) node[anchor=north east] {$0$};
%        \foreach \i in {-3,-2,-1,1,2,3}
%            {
%            \draw[fill] (0,\i ) circle (1pt) node[anchor=east] {$\i$};
%            \draw[fill] (\i,0) circle (1pt) node[anchor=north] {$\i$};
%            \foreach \j in {-3,-2,-1,1,2,3}
%                \draw[fill] (\j, \i) circle (1pt);
%            }
%        %lines separating sectors
%        \draw[thick, blue] (0.5, -3.5) -- (0.5, 3.5);
%        \draw[thick, blue] (-3.5, 0.5) -- (3.5, 0.5);
%
%        \node at (2, 2.5) {$\bm{\theta_I} = \{1, 1\}$};
%        \node at (-2, 2.5) {$\bm{\theta_{II}} = \{0, 1\}$};
%        \node at (2, -2.5) {$\bm{\theta_{III}} = \{1, 0\}$};
%        \node at (-2, -2.5) {$\bm{\theta_{IV}} = \{0, 0\}$};
%
%        \clip (-3.3,-3.3) rectangle (0.3, 0.3);
%        \foreach \i in {-12,...,0}
%            \draw[dashed, red, samples=100] plot function{-x+(0.3+0.5*\i)} node[right] {};
%    \end{tikzpicture}
%    \caption{Each point $(\nu_1, \nu_2)$ in the $\mathbb{Z}^2$ lattice corresponds to the $d$-dimensional Feynman integral $I(\nu_1, \nu_2)$. The lattice is divided into sectors as defined through Eq.~\ref{eq:sectors}. They are ordered $\bm{\theta}_{I}>\bm{\theta}_{II},\,\bm{\theta}_{III}>\bm{\theta}_{IV}$. In particular, sector $\bm{\theta}_{IV}$ is trivially 0.} 
%    \label{fig:IBPsectors}
%    \end{subfigure}
%    \hfill
%    \hspace*{1em}\raisebox{0.9cm}{ % needed to align the diagrams
%    \begin{subfigure}[b]{0.45\textwidth}
%    \begin{tikzpicture}
%      %axes
%        \draw[thick,->] (-0.5,0) -- (5.5,0) node[anchor=north west] {$\nu_1$};
%        \draw[thick,->] (0,-0.5) -- (0,6.5) node[anchor=south east] {$\nu_2$};
%
%        %dots and axes labels
%        \draw[fill] (0,0) circle (1pt) node[anchor=north east] {$0$};
%        \foreach \i in {1,...,6}
%            {
%            \draw[fill] (0,\i ) circle (1pt) node[anchor=east] {$\i$};
%            \draw[fill] (\i,0) circle (1pt) node[anchor=north] {$\i$};
%            \foreach \j in {1,...,5}
%                \draw[fill] (\j, \i) circle (1pt);
%            }
%
%        %lines separating sectors
%        \draw[thick, blue] (0.5, -0.5) -- (0.5, 6.5);
%        \draw[thick, blue] (-0.5, 0.5) -- (5.5, 0.5);
%
%        \draw[orange, thick, ->] (5, 4) edge (5,3) (5, 3) edge (4,3) (4,3) edge (4,2) (4,2) edge (3,2) (3,2) edge (2,2) (2,2) edge (2,1) (2, 1) -- (1,1);
%        
%        \node at (5, 6.5) {$\bm{\theta_I}$};
%        \node at (1, 1) [anchor=south] {$\text{MI}(1, 1)$};
%        \node at (5, 4) [anchor=south] {$I(5, 4)$};
%    \end{tikzpicture}
%    \caption{A hypothetical reduction pathway within the top sector $\bm{\theta}_I$. An integral $I(5, 4)$ is reduced to the master integral $\text{MI}(1,1)$ through a series of IBP relations lowering the denominator exponents.} \label{fig:IBPreduction}
%    \end{subfigure}
%    }
%    \caption{A visualisation of integrals and sectors used in the IBP reduction as a lattice of points ($N=2$).}
%    \label{fig:IBPschematic}
%\end{figure}

%\begin{figure}
%    \centering
%        \begin{tikzpicture}
%        	\begin{feynman}
%        		\vertex (v1);
%          
%        		\vertex[right = of v1] (v2);			
%        		\vertex[below = of v2] (v3);			
%        		\vertex[left = of v3] (v4);			
%          
%        		\vertex[above left = 0.7cm of v1] (i1) {$p_1$};
%        		\vertex[above right =  0.7cm of v2] (i2) {$p_2$};
%        		\vertex[below right =  0.7cm of v3] (i3) {$p_3$};
%        		\vertex[below left =  0.7cm of v4] (i4) {$p_4$};
%
%                \node (dot) at ($(v2)!0.5!(v3)$);
%                \draw[fill] (dot) circle (2pt);
%          
%        		\diagram*{
%                    (v1) -- (v2) -- (v3) -- (v4) -- [momentum=$k_1$] (v1),
%                    (i1) -- (v1),
%                    (i2) -- (v2),
%                    (i3) -- (v3),
%                    (i4) -- (v4),
%                    };
%        	\end{feynman}
%        \end{tikzpicture}
%    \caption{One-loop box Feynman diagram $I_\text{box}(1,1,2,1)$ with a `dotted' propagator. The corresponding denominators are: $\{k_1^2, (k_1-p_1)^2, \left((k_1-p_1-p_2)^2\right)^2, (k_1+p_4)^2\}$\,.} \label{fig:1Lboxdot}
%\end{figure}

%\begin{figure}
%    \centering
%    \begin{adjustbox}{minipage=\textwidth,scale=0.9}
%    \raisebox{0.2cm}{
%    \begin{subfigure}[b]{0.25\linewidth}
%        \hspace*{1em}
%        \begin{tikzpicture}
%    	\begin{feynman}[small]
%    		\vertex (v1);
%    		\vertex[above left = 0.8cm of v1] (i1) {$p_5$};
%    		\vertex[right = of v1] (v2);
%    		\vertex[yshift=0.3cm, right = 0.8cm of v2] (v3);
%    		\vertex[above right = 0.8cm of v3] (f1) {$p_1$};			
%    		
%    		\vertex[below = of v1] (v7) ;
%    		\vertex[below left = 0.8cm of v7] (i2) {$p_4$};
%    		\vertex[right = of v7] (v6) ;
%    		\vertex[yshift=-0.3cm, right = 0.8cm of v6] (v5) ;
%    		\vertex[below right = 0.8cm of v5] (f3) {$p_3$};
%    		
%    		\vertex[xshift=0.6cm, yshift=0.2cm, below = of v3] (v4);
%    		\vertex[right = 0.8cm of v4] (f2) {$p_2$};		
%
%            \draw ($(v7)!0.5!(v6)$) node[cross,red] (5cm) {};
%            \draw ($(v4)!0.5!(v5)$) node[cross,red,rotate=45] (5cm) {};
%            
%    		\diagram*{
%    			(i1) -- (v1) -- (v2) -- (v3),
%    			(i2) -- (v7) -- (v6) -- (v5),
%    			(v3) -- (f1),
%    			(v3) -- (v4) -- (v5),
%    			(v4) -- (f2),
%    			(v5) -- (f3),
%    			(v1) -- (v7),
%    			(v2) -- (v6),
%    		};
%    	\end{feynman}
%    \end{tikzpicture}
%    \caption{} \label{fig:topologyc}
%    \end{subfigure}
%    }
%    %\end{adjustbox}
%    \hspace{1.3cm}
%    %\begin{adjustbox}{minipage=\linewidth,scale=0.4}
%    \begin{subfigure}[b]{0.25\linewidth}
%        \hspace*{1em}
%        \begin{tikzpicture}
%        \begin{feynman}[small]
%			\vertex (v1);
%			\vertex[left = 0.8cm of v1] (i1) {$p_5$};
%			\vertex[above right =1cm of v1] (v2);
%			\vertex[below right =1cm of v1] (v3);				
%			\vertex[yshift=0.3cm, right = 0.7cm of v2] (v4);	
%			\vertex[yshift=-0.6cm, right = 0.7cm of v4] (v5);
%			\vertex[yshift=-0.3cm, right = 0.7cm of v3] (v7);
%			\vertex[yshift=0.6cm, right = 0.7cm of v7] (v6);	
%			\vertex[above right = 0.8cm of v4] (i2) {$p_1$};
%			\vertex[yshift=0.2cm, right = 0.8cm of v5] (i3) {$p_2$};
%			\vertex[yshift=-0.2cm, right = 0.8cm of v6] (i4) {$p_3$};
%			\vertex[below right = 0.8cm of v7] (i5) {$p_4$};
%			
%            \draw ($(v7)!0.5!(v3)$) node[cross,red,rotate=60] (5cm) {};
%            \draw ($(v6)!0.5!(v5)$) node[cross,red] (5cm) {};
%            
%			\diagram*{
%				(i1) -- (v1) -- (v2) -- (v3) -- (v1),
%				(v2) -- (v4) -- (v5) -- (v6) -- (v7) -- (v3),
%				(v4) -- (i2),
%				(v5) -- (i3),
%				(v6) -- (i4),
%				(v7) -- (i5),
%			};
%		\end{feynman}
%        \end{tikzpicture}
%    \caption{} \label{fig:topologyc}
%    \end{subfigure}
%    %\end{adjustbox}
%    \hspace{1.3cm}
%    %\begin{adjustbox}{minipage=\linewidth,scale=0.4}
%    \raisebox{0.2cm}{
%    \begin{subfigure}[b]{0.25\linewidth}
%        \hspace*{1em}
%        \begin{tikzpicture}
%    	\begin{feynman}[small]
%    		\vertex (v1);
%      
%    		\vertex[left = 0.8cm of v1] (i1) {$p_5$};
%    		\vertex[above right = of v1] (v2);			
%    		\vertex[below right = of v1] (v3);			
%    		\vertex[right = of v2] (v4);			
%    		\vertex[right = of v3] (v5);			
%    		
%    		\vertex[above right = 0.8cm of v4] (i2) {$p_1$};
%    		\vertex[below right = 0.7cm of v5, yshift=-0.2cm] (i3) {$p_3$};
%    		\vertex[below right = 0.5cm of v5, xshift=+0.4cm] (i4) {$p_2$};
%    		\vertex[below =  0.7cm of v3] (i5) {$p_4$};
%    		
%    		\diagram*{
%                (i1) -- (v1) -- (v2) -- (v4) -- (i2),
%                (v1) -- (v3) -- (v5) -- (i3),
%                (v2) -- (v3),
%                (v4) -- (v5),
%                (v5) -- (i4),
%                (v3) -- (i5),
%    		};
%    	\end{feynman}
%    \end{tikzpicture}
%    \caption{} \label{fig:topologyb}
%    \end{subfigure}
%    }
%    \end{adjustbox}
%\caption{A simple example of possible non-IBP relations between integrals within different (maximal) topologies. Integrals in lower sectors often map to each other. The red crosses denote a vanishing denominator, i.e. the corresponding $\nu_i$ is 0. Here, diagrams $(a)$ and $(b)$ both collapse onto diagram $(c)$. Such relations are often hard to discover just from analysing the propagators of the integrals, especially if the two families use vastly different conventions for loop momentum routing. However, a visual representation often makes spotting these relations trivial.}
%\label{fig:topologies}
%\end{figure}


\begin{table}[b]
	\begin{center}
		\begin{tabular}{|c|c|c|c|c|}
            \hline
            External masses & Type & Topology & Publication & Notes \\
			\hline
            \multirow{2}{0cm}{0} & planar & penta-box & \cite{Gehrmann:2015bfy,Papadopoulos:2015jft, Gehrmann:2018yef} & See also \cite{Chicherin:2020oor} \\
            \cline{2-5}
            & \multirow{2}{2cm}{non-planar} & hexa-box & \cite{Chicherin:2018mue} & See also \cite{Chicherin:2017dob, Chicherin:2018ubl, Chicherin:2018wes, Abreu:2018rcw, Chicherin:2020oor} \\
            & & double-pentagon & cc & dd \\
            \hline
            \multirow{2}{0cm}{1} & planar \\
            \cline{2-5}
            & non-planar \\
            \hline
		\end{tabular}
\end{center}
\end{table}

\end{document}